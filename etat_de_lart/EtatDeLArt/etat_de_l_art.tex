\chapter{Etat de l'art}
\section{Introduction}
\paragraph{}
Une étude ciblée, approfondie et critique des travaux (existants) a été réalisée sur 
les GIS. Elle a permis la maîtrise du domaine de recherche par l’acquisition des 
connaissances solides sur les travaux de recherche réalisés.

\section{situation actuelle d'Haiti}
Lorem ipsum dolor sit amet, consectetur adipiscing elit. Mauris at ultrices purus. Donec finibus metus et augue sodales posuere. Proin sit amet turpis dictum, iaculis felis in, scelerisque massa. Nullam aliquam nunc eget fringilla volutpat. Integer et mauris et massa imperdiet scelerisque mollis at sapien. Donec condimentum felis eget sagittis ultricies. Nunc laoreet augue id consectetur vulputate. Cras sagittis aliquam risus sit amet tempus. Curabitur finibus neque eget magna efficitur, sed dignissim quam sagittis. Ut euismod justo id gravida pulvinar. Ut urna magna, auctor maximus volutpat ac, elementum sed mi.


\section{Standards et modèles existants}
Lorem ipsum dolor sit amet, consectetur adipiscing elit. Mauris at ultrices purus. Donec finibus metus et augue sodales posuere. Proin sit amet turpis dictum, iaculis felis in, scelerisque massa. Nullam aliquam nunc eget fringilla volutpat. Integer et mauris et massa imperdiet scelerisque mollis at sapien. Donec condimentum felis eget sagittis ultricies. Nunc laoreet augue id consectetur vulputate. Cras sagittis aliquam risus sit amet tempus. Curabitur finibus neque eget magna efficitur, sed dignissim quam sagittis. Ut euismod justo id gravida pulvinar. Ut urna magna, auctor maximus volutpat ac, elementum sed mi.


\section{Outils de développement }
\paragraph{}
Il existe de nombreux outils de développement de vues de supervision. Ceux-ci sont
basés sur des technologies diverses et ont chacun leurs propres avantages et
inconvénients par rapport à notre problématique. Notre principal problème est le 
protocole permettant les échanges des donées géotechniques.

\section{Processus de développement}
\paragraph{}
La figure représente le processus de développement centré architecture ainsi que
les acteurs qui interviennent directement dans celui-ci : l’architecte, le développeur, ainsi
qu’éventuellement l’analyste (cette tâche peut être automatisée) [Leymonerie 2004]. 
L’architecte a pour rôle de définir l’architecture qui servira de base au développement
de l’application. Le développeur raffine cette architecture de façon à s’approcher petit à
petit de l’application finale. Pour cela il implémente les composants ainsi que leurs
interactions (les connecteurs) en respectant la structure et les propriétés définies par
l’architecture de départ* \par 
A chaque étape de raffinement l’analyste est en mesure de
vérifier que l’architecture raffinée est conforme à l’architecture du niveau d’abstraction
supérieur. Ce processus permet de garantir que l’application obtenue respecte les
propriétés fonctionnelles, structurelles et comportementales définies par l’architecte en
accord avec le client et les utilisateurs.