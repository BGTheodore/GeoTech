\section{Resume}
\paragraph{}
La gestion des données géothechniques en Haïti ne dispose pas d’un système informatisé centralisé. Dans ce mémoire, nous vous présentons une solution se basant sur la conception et la réalisation d'un tel systeme.\par 

La totalité des codes est disponible en ligne sur GitHub :https://github.com/geotech
L’application est hébergé par AWS à l’adresse: https://geotech.ht

\paragraph{Abstract}
\par 
The sketch is an old form of communication that was used to visualize, share and store information. Despite its proven
expressiveness, it has not yet become a used modality in the interaction between humans and computer systems. Geographic
Information Systems (GIS) have special needs for such advanced forms of interaction, because they involve complex and
heterogeneous data structures, which are often difficult to describe by the text or by predicates based on attributes. The objective of
this work is to show the usefulness of the sketch in retrieval spatial information in a particular case which is the geographic data. In
this context, a prototype application of sketch query system is implemented in order to verify the concepts and theories developed.