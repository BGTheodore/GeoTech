\chapter*{Résumé}
\addcontentsline{toc}{chapter}{Résumé}  
\paragraph{}
Ce travail s'inscrit dans le cadre du projet \textbf{Kay Nou Tek}
financé par l’Ambassade de Suisse en Haïti. 
L'Unité de Recherche en Géoscience (URGéo) a pour objectif de concevoir et de réaliser
une application web permettant la mise en 
ligne gratuite de certaines informations contenues dans les tests 
géotechniques réalisés par elle et d’autres laboratoires.
Par ailleurs, elle sera responsable de la collecte des informations, de leur tri, de l’alimentation 
de l’application et de sa gestion. 
\paragraph{}
La première phase du projet consiste à réaliser la base de données géotechniques
. Cette dernière permettra de mutualiser les données 
 sur le sous-sol haïtien accumulées par différents organismes, tant publiques que privées, 
 au cours de ces 50 dernières années. Ensuite il sera question d'exploiter ces 
 données dans un Système d'Information Géographique (SIG)  afin de faciliter leur visualisation 
 dans une application
 Web\footnote{ 
    La totalité des codes est disponible en ligne sur GitHub : 
    \url{https://github.com/geotech}.
    L’application est accessible via l’adresse: 
    \url{https://geotech.ht}
 }. Celle-ci
 constituera donc un espace 
 partagé permettant à ces différents organismes de centraliser et de gérer une banque 
 de données géotechniques.


\newpage
\chapter*{Abstract}
\addcontentsline{toc}{chapter}{Abstract}  
\paragraph{}
This work is part of the Kay Nou Tek project funded by the Swiss 
Embassy in Haiti. The objective of the Unité de Recherche en Géoscience (URGéo) 
is to design and produce an application allowing the free online posting 
of certain information contained in the geotechnical tests carried out by 
it and other laboratories. In addition, she will be responsible for collecting 
information, sorting it, feeding the application and managing it.
\paragraph{}
The first phase of the project consists of creating the geotechnical database
. The latter will allow data to be pooled
 on the Haitian subsoil accumulated by various organizations, both public and private,
 over the past 50 years. Then it will be a question of exploiting these
 data in a Geographic Information System (GIS) in order to facilitate their visualization in an application.
  This Web application\footnote{ 
    All the codes are available online on GitHub: 
    \url{https://github.com/geotech}.
    The application is accessible via the address: 
    \url{https://geotech.ht}
    }
  will therefore constitute a space
 shared allowing these different organizations to centralize and manage a bank
 geotechnical data.

