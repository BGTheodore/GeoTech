\chapter*{Introduction}
Depuis la naissance de l'informatique géologique (an-
nie 1970), toutes les études qui ont été faites dans ce
domaine ont été orientées vers la recherche des
méthodes les plus fiables et les moins coûteuses pour
stocker, traiter et restituer les données géologiques et
géotechniques concernant les 3 sujets suivants :
\paragraph{programmes de base de données (ou banque de
données) }: dont l'objectif est le stockage des donn6es;

\paragraph{programmes de calculs}
 dont l'objectif est le traite-
ment mathématique, statistique et éventuellement géo-
statistique des données;

\paragraph{programrnes de cartographie}
 dont l'objectif est la
restitution des données géologiques et géotechniques
sous forme graphique.
\par
Historiquement, le stockage des données géologiques
et géotechniques a été fait sur des fichiers séquentiels
spécialisés dont chacun ne présente qu'une application.
\par
La redondance des données (répétition des mêmes
données dans des fichiers différents) était é1evée, et il
n'y avait pas une indépendance entre les données et les
programmes de traitement. I1 était aussi nécessaire de
codifier de faqon externe et interne les données géologiques pour les stocker. C'est seulement en utilisant
des systèmes de gestion de base de données généraux
qu'on a pu diminuer la redondance entre les données
et assurer l'indépendance des programmes.
\par
 Au début,
les auteurs ont considéré que les données de notre
domaine pouvaient avoir une structure hiérarchique
(base de données arborescente). A la suite de l'échec
de ce type de base de données, on a travaillé sur des
bases de données en réseau. Ces deux types de base de
données n'ont pas pu éliminer d'une faqon définitive
la redondance et surtout la codification interne des
données géologiques.
\cite{tunis}
\par
Pour atteindre cet objectif, assurer une indépendance
totale entre les données et les programmes de traitement et obtenir une base de données géologiques et
géotechniques très souple, nous utilisons, pour la
conception de << GEO-TECH>>, le système
de gestion de base de données relationnel général en
considérant que toutes les données pouvaient être
modé1isées sous forme de tableaux  dont
l'une des colonnes est la clef. Chaque colonne sera un
domaine géologique ou géotechnique et chaque ligne (tuple)
un enregistrement.