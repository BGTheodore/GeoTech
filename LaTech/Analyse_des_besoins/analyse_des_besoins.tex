\chapter{Analyse des besoins}
        Nous débutons la conception de notre système en analysant la
situation pour prendre note des différentes contraintes, des risques
et tout autre élément pertinent dans le but de satisfaire l'intégralité
des besoins de l'URGéo.  Nous sommes déjà imbus du contexte de développement
du système, par conséquent, nous allons, dans cette partie, nous concentrer
sur les besoins et les contraintes de l'application.
\section{Besoins et contraintes}
        Il s'agit de la conception d'une base de données géotechniques et d'une
        application web permettant de visualiser cesdites données. Définissons
        d'abord tous les besoins des différents utilisateurs du système.
        \subsubsection{Identification des acteurs du système}
        Pour connaître les différents besoins des utilisateurs, nous devons
        avant tout relever la liste des différents utilisateurs eux-mêmes.
        Nombreux sont ceux qui auront à utiliser le système. Nous appellerons ces différents
        utilisateurs  les \textbf{acteurs} du système.
        \par
        L'application est disponible pour tout le monde notamment les
        professionnels en géosciences, les ingénieurs, les étudiants, les banques, les 
        compagnies d'assurance, etc.
        Ces acteurs sont divisés en trois (3) catégories:
        \begin{itemize} 
                \item \textbf{les visiteurs: }
                Un visiteur est un utilisateur externe qui se rend sur l'application pour rechercher et visualiser
                les données mises disponibles par l'URGéo et les instances associées. 
              
                \item \textbf{les administrateurs: }
                Un administrateur est un utilisateur interne capable d'interagir directement avec les données. 
                Il a pour principal rôle la gestion des data 
                liés aux différents résultats géotechniques. Tout au long du document, on entendra par data:
                \begin{itemize}
                        \item Les résultats des essais 
                        \item Les fonds de carte
                \end{itemize}
                Il est obligatoire pour lui de s'authentifier pour pouvoir 
                effectuer certaines actions sur le système. Seront administrateurs, toute personne désignée par l'URGéo
                ou les partenaires de l'URGéo. Le plus souvent, il s'agira des stagiares responsables de l'entrée 
                des données.

                \item \textbf{les superadmins: }
                Un super-administrateur est un super utilisateur. Détenant un rôle particulièrement sensible,
                il est obligatoire pour lui de s'authentifier pour y accéder. Ses droits sont particulièrement orientés 
                vers la gestion des ressources du système. Il peut ainsi manipuler les informations relatives aux 
                différents utilisateurs et garder un trace des trafics effectués au sein de l'application.
                 Seront superadmins, toute personne désignée par l'URGéo.
            \end{itemize}   
        \subsubsection{Besoins des différents utilisateurs}
        Étant donné que l'on a deux types d'utilisateurs internes avec des privilèges différents,
        le système doit impérativement comporter un mode de gestion des utilisateurs et des droits d'accès.
        \paragraph{Le visiteur}
        Le visiteur a à sa disposition une carte d'Haïti marquée aux différents endroits où des tests 
        géotechniques ont été réalisés.
        À n'importe quel moment, il peut décider d'effectuer une recherche par mot clé et s'attend
        à ce que le résultat de sa recherche s'affiche sur la carte. Il a aussi l'option de l'afficher sous la forme
        d'une liste qu'il peut filtrer selon son choix. Cette dernière peut être téléchargée sous format CSV.
        En support aux informations spécifiques à un test se trouvant à un endroit bien précis sur la carte,
        le visiteur a aussi l'accès au résultat du test se trouvant dans un fichier PDF qu'il peut télécharger.
        \par
        Aussi, plusieurs fonds de carte seront disponibles permettant au visiteur d'adapter le résultat de ses recherches
        au contexte idéal (topographie, hydraulique,... )
        \par
        Le visiteur peut aussi décider de lire, de commenter ou de laisser un message (de manière anonyme ou pas) 
        sur le forum dédié à l'application.
        \paragraph{L'administrateur}
        Un administrateur ne peut exister sans appartenir à une institution su système. Avant tout, il peut réaliser 
        toutes les actions d'un visiteur. De plus, après s'être authentifié au moyen de 
        son adresse électronique et de son mot de passe, il peut interagir directement avec la base de données. En cas 
        d'oubli de son mot de passe, le système lui envoie un lien de réinitialisation de mot de passe à son email.
        Pour jouer son rôle d'administrateur, il est redirigé vers \textit{l'interface de l'administrateur}. 
        Dans ce module, l'administrateur peut:
        \begin{itemize}
                \item \textbf{Ajouter un test: }
                Il s'agit de rentrer les informations relatives à un test pour l'ajouter dans la base de données.
                Ces informations sont de types différents (nom:texte, identifiant:entier, date du test:date, types
                de test:entier, date d'enregistrement:date, etc\footnote{Les différents champs et leur type seront 
                détaillés dans l'étude des diagrammes à la fin du chapitre} )
                \item \textbf{Modifier un test: }
                Si pour une raison ou pour une autre un test doit être modifié, l'administrateur est en
                mesure de le faire après s'être authentifié. Un message lui sera affiché à l'écran dépendemment 
                de la réussite ou de l'échec de son action. 
                \item \textbf{Supprimer un test: }
                La suppression d'un test est aussi possible. Un message de confirmation précède la validation
                de l'exécution de cette action car elle est irréversible.
        \end{itemize}
        \par
        À noter qu'il ne peut s'aventurer à modifier ou supprimer un test qui n'avait pas été directement 
        ajouté par un administrateur appartenant à la même institution. De plus, si l'URGéo juge que le commentaire d'un visiteur doit être supprimé,
        l'administrateur est apte à réaliser cela.
        \par
        Chaque action effectuée par un administrateur sera enregistrée automatiquement pour permettre la traçabilité
        et la non-répudiation\footnote{On abordera cette partie dans la section sécurité du chapitre 3.}.
        Ainsi, un module permettant de visualiser uniquement les logs\footnote{Historique des actions effectuées sur un 
        système informatique.} du système. Par conséquent, on peut savoir
        la date et l'heure précise où un administrateur ouvre une session, affiche, ajoute, modifie ou supprime une donnée.
        Nul utilisateur ne pourra altérer ces donnéees.
        \par

        \paragraph{Le superadmin}
        Il s'agit là de l'utilisateur de plus haute hiérarchie de notre application. Certes, il est libre 
        d'utiliser l'application comme un simple visiteur. De plus, après s'être authentifié au moyen de 
        son adresse électronique et de son mot de passe, il peut avoir accès tant aux données relatives aux
        différents utilisateurs de son institution qu'au trafic des différentes données en circulation. Il 
        pourra ainsi visionner les statistiques relatives au bien fondé de la plateforme.
        Dans ce module, le superadmin peut:
        \begin{itemize}
                \item \textbf{Ajouter un utilisateur: }
                Il s'agit de rentrer les informations relatives à un utilisateur pour l'ajouter dans la base de données.
                Ces informations sont de types différents (nom:texte, prénom identifiant:entier, type d'utilisateur, 
                etc\footnote{Les différents champs et leur type seront 
                détaillés dans l'étude des diagrammes à la fin du chapitre} )
                \item \textbf{Modifier un utilisateur: }
                Si pour une raison ou pour une autre les informations d'un utilisateur doivent être modifiées, le superadmin est en
                mesure de le faire après s'être authentifié. Un message lui sera affiché à l'écran dépendemment 
                de la réussite ou de l'échec de son action.
                \item \textbf{Activer ou désactiver un utilisateur: }
                Il s'agit d'autoriser ou non un administrateur à utiliser l'application.
                \item \textbf{Visionner la statistique des trafics effectués sur les données de son institution: }
                Un graphe statistique sera mis à sa disposition sans qu'il ne puisse le modifier personnellement.
        \end{itemize}
          

\par    
\begin{table}
        \centering
        \begin{tabular}{|p{0.21\linewidth}|p{0.54\linewidth}|p{0.33\linewidth}|}
        \hline
                \textbf{Utilisateurs} & \textbf{Besoins} & 
                \textbf{Contraintes}  \\
                \hline
                        Visiteur & 
                        \begin{itemize}
                                 \item[$\cdot$]  Cartographie d'Haïti
                                 \item[$\cdot$]  Fonds de carte
                                 \item[$\cdot$]  Recherches
                                 \item[$\cdot$]  Filtrage des donnéees
                                 \item[$\cdot$]  Téléchargement des résultats des tests
                                 \item[$\cdot$]  Navigation simple et attrayante
                        \end{itemize} & 
                        Accès au site à partir du lien \\
                \hline
                        Administrateur & 
                        \begin{itemize}
                                \item[$\cdot$]  Ajout de test
                                \item[$\cdot$]  Modification de test
                                \item[$\cdot$]  Suppression de test
                                \item[$\cdot$]  Suppression de commentaire
                        \end{itemize} & 
                        \begin{itemize}
                                \item[$\cdot$] Authentification 
                                \item[$\cdot$] Appartenance à une institution du système 
                        \end{itemize}
                         \\
                \hline
                        Superadmin & 
                        \begin{itemize}
                                \item[$\cdot$]  Affichage des logs
                                \item[$\cdot$]  Ajout d'administrateur
                                \item[$\cdot$]  Modification d'administrateur
                                \item[$\cdot$]  Activation d'administrateur
                                \item[$\cdot$]  Désactivation d'administrateur
                                \item[$\cdot$]  Accès à l'analyse des flux de l'application 
                        \end{itemize} & 
                        \begin{itemize}
                                \item[$\cdot$] Authentification 
                                \item[$\cdot$] Appartenance à une institution du système 
                        \end{itemize}
                         \\
                \hline  
        \end{tabular}
        \caption{Tableau des utilisateurs et de leurs besoins} \label{tab:sometab}
\end{table}
\par
                \lipsum[1]
        \section{Approche de travail}
                \lipsum[1]
        \section{Méthodologie}
                \lipsum[1]
        \section{Structure modulaire}
                \lipsum[1]
        \section{Préparation des documents}
                \par
                Étant donnée que la majorité des rapports et résultats des tests sont 
                disponibles sous forme papier, la première étape a consisté à scanner les documents.
                Pour ce faire nous adoptons une protrocole: on assure la traćabilité de chaque 
                document les munissant d'une cartouche reprennant une série d'informations.
                Voici la liste des informations qui forme une cartouche:
                \begin{itemize}
                        \item \textbf{Lieu: }
                        Il s'agit de l'endroit où l'étude a été effectuée.
                        \item \textbf{Date: }
                        Il s'agit de la date à laquelle l'étude a été effectuée.
                        \item \textbf{Maître d'ouvrage: }
                        Il s'agit du client pour lequel l'étude est effectuée.
                        \item \textbf{Maître d'œuvre : }
                        Il s'agit de a personne ou l'entreprise chargée de l'étude.
                        \item \textbf{Réf Numéro Étude: }
                        Il s'agit d'un identifiant unique permettant de tracer une étude.
                \end{itemize}
                
\par    
\begin{table}
        \centering
        \begin{tabular}{|p{0.30\linewidth}|p{0.40\linewidth}|}
                \hline
                \textbf{Lieu} & Delmas \\
                \hline
                \textbf{Date} & Janier 2020 \\
                \hline
                \textbf{Maître d'ouvrage} & Faculté Des Sciences \\
                \hline
                \textbf{Maître d'œuvre} & URGéo \\
                \hline
                \textbf{Réf Numéro Étude} & 1234 \\
                \hline
        \end{tabular}
        \caption{Exemple de cartouche sur un document d'étude géotechnique} \label{tab:example_cartouche}
\end{table}
\par

        \section{Modélisation avec UML}
                \lipsum[1]
