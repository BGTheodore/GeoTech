\chapter{Analyse des besoins}
        Nous débutons la conception de notre système en analysant la
situation pour prendre note des différentes contraintes, des risques
et tout autre élément pertinent dans le but de satisfaire l'intégralité
des besoins de l'URGéo.  Nous sommes déjà imbus du contexte de développement
du système, par conséquent, nous allons, dans cette partie, nous concentrer
sur les besoins et les contraintes de l'application.
\section{Besoins et contraintes}
        Il s'agit de la conception d'une base de données géotechniques et d'une
        application web permettant de visualiser cesdites données. Définissons
        d'abord tous les besoins des différents utilisateurs du système.
        \subsubsection{Identification des acteurs du système}
        Pour connaître les différents besoins des utilisateurs, nous devons
        avant tout relever la liste des différents utilisateurs eux-mêmes.
        Nombreux sont ceux qui auront à utiliser le système. Nous appellerons ces différents
        utilisateurs  les \textbf{acteurs} du système.
        \par
        L'application sera disponible pour tout le monde notemment les
        professionnels en géosciences, les ingénieurs, les étudiants, les banques, les 
        compagnies d'assurance, etc.
        Ces acteurs sont divisés en trois (3) catégories:
        \begin{itemize} 
                \item \textbf{les visiteurs: }
                Un visiteur est un utilisateur lambda qui se rend sur l'application pour rechercher et visualiser
                les données mises disponibles par l'URGéo et les instances associées. 
              
                \item \textbf{les administrateurs: }
                Un administrateur est un utilisateur avec plus de privilèges. Il a non seulement les droits d'un visiteur,
                mais aussi d'autres attributions. Il est obligatoire pour lui de s'authentifier pour pouvoir 
                effectuer certaines actions sur les systèmes. Seront administrateurs, toute personne désignée par l'URGéo
                ou les partenaires de l'URGéo. Le plus souvent, ce seront les stagiares responsables de l'entrée des données.

                \item \textbf{les superadmins: }
                Un super-administrateur est un super utilisateur. Il a non seulement les droits d'un visiteur,
                ceux de l'administrateur mais aussi d'autres attributions. Il est obligatoire pour lui de s'authentifier pour pouvoir 
                effectuer certaines actions sur les systèmes. Seront superadmins, toute personne désignée par l'URGéo.
            \end{itemize}   
        \subsubsection{Besoins des différents utilisateurs}
        Étant donné que l'on aura deux types d'utilisateurs avec des privilèges différents,
        le système doit impérativement comporter un mode de gestion des utilisateurs et des droits d'accès.
        \paragraph{Le visiteur}
        Le visiteur a à sa disposition une carte d'Haïti marquée aux différents endroits où des tests 
        géotechniques ont été réalisés.
        À n'importe quel moment, il peut décider d'effectuer une recherche par mot clé et s'attend
        à ce que le résultat de sa recherche s'affiche sur la carte. Il a aussi l'option de l'afficher sous la forme
        d'une liste, il a la possibilité de filtrer selon son choix. Cette dernière peut être téléchargée sous format CSV.
        En support aux informations spécifiques à un test se trouvant à un endroit bien précis sur la carte,
        le visiteur a aussi l'accès au résultat du test se trouvant dans un fichier PDF qu'il peut télécharger.
        \par
        Aussi, plusieurs fonds de carte seront disponibles permettant au visiteur d'adapter le résultat de ses recherches
        au contexte idéal (topographie, hydraulique,... )
        \par
        Le visiteur peut aussi décider de lire, de commenter ou de laisser un message (de manière anonyme ou pas) sur le forum dédié à l'application.
        \paragraph{L'administrateur}
        Avant tout, il peut réaliser toutes les actions d'un visiteur. De plus, après s'être authentifié au moyen de 
        son adresse électronique et de son mot de passe, il peut interagir directement avec la base de données. En cas 
        d'oubli de son mot de passe, le système lui envoie un lien de réinitialisation de mot de passe à son email.
        Pour jouer son rôle d'administrateur, il est redirigé vers \textit{l'interface de l'administrateur}. 
        Dans ce module, l'administrateur peut:
        \begin{itemize}
                \item \textbf{Ajouter un test: }
                Il s'agit de rentrer les informations relatives à un test pour l'ajouter dans la base de données.
                Ces informations sont de types différents (nom:texte, identifiant:entier, date du test:date, types
                de test:entier, date d'enregistrement:date, etc\footnote{Les différents champs et leur type seront 
                détaillés dans l'étude des diagrammes à la fin du chapitre} )
                \item \textbf{Modifier un test: }
                Si pour une raison ou pour une autre un test doit être modifié, l'administrateur est en
                mesure de le faire après s'être authentifié. Un message lui sera affiché à l'écran dépendemment 
                de la réussite ou de l'échec de son action.
                \item \textbf{Supprimer un test: }
                La suppression d'un test est aussi possible. Un message de confirmation précède la validation
                de l'exécution de cette action car elle est irréversible.
        \end{itemize}
        \par
        De plus, si l'URGéo juge que le commentaire d'un visiteur doit être supprimé,
        l'administrateur est apte à réaliser cela.
        \par
        Chaque action effectuée par un administrateur sera enregistrée automatiquement pour permettre la traċabilité
        et la non-répudiation\footnote{On abordera cette partie dans la section sécurité du chapitre 3.}.
        Ainsi, un module permettant de visualiser uniquement les logs\footnote{Historique des actions effectuées sur un système informatique.} du système. Par conséquent, on peut savoir
        la date et l'heure précise où un administrateur ouvre une session, affiche, ajoute, modifie ou supprime une donnée.
        Nul utilisateur ne pourra altérer ces donnéees.
        \par

        \paragraph{Le superadmin}
        Avant tout, il peut réaliser toutes les actions d'un visiteur. De plus, après s'être authentifié au moyen de 
        son adresse électronique et de son mot de passe, il peut interagir directement avec la base de données et 
        effectuer aussi les mêmes actions que l'administrateur.
        Dans ce module, le superadmin peut:
        \begin{itemize}
                \item \textbf{Ajouter un utilisateur: }
                Il s'agit de rentrer les informations relatives à un utilisateur pour l'ajouter dans la base de données.
                Ces informations sont de types différents (nom:texte, prénom identifiant:entier, type d'utilisateur, 
                etc\footnote{Les différents champs et leur type seront 
                détaillés dans l'étude des diagrammes à la fin du chapitre} )
                \item \textbf{Modifier un utilisateur: }
                Si pour une raison ou pour une autre les informations d'un utilisateur doivent être modifiées, le superadmin est en
                mesure de le faire après s'être authentifié. Un message lui sera affiché à l'écran dépendemment 
                de la réussite ou de l'échec de son action.
                \item \textbf{Activer ou désactiver un utilisateur: }
                Il s'agit d'autotoriser ou non un administrateur à utiliser l'application.
        \end{itemize}
          

\par    
\begin{table}
        \centering
        \begin{tabular}{|p{0.21\linewidth}|p{0.54\linewidth}|p{0.21\linewidth}|}
        \hline
                \textbf{Utilisateurs} & \textbf{Besoins} & 
                \textbf{Contraintes}  \\
                \hline
                        Visiteur & 
                        \begin{itemize}
                                 \item[$\cdot$]  la cartographie d'Haïti
                                 \item[$\cdot$]  les fonds de carte
                                 \item[$\cdot$]  faire des recherches
                                 \item[$\cdot$]  de filtrer les donnéees
                                 \item[$\cdot$]  télécharger les résultats des tests
                                 \item[$\cdot$]  une navigation simple et attrayante
                        \end{itemize} & 
                         \\
                \hline
                        Administrateur & 
                        \begin{itemize}
                                \item[$\cdot$]  tous les besoins du visiteur
                                \item[$\cdot$]  ajouter un test
                                \item[$\cdot$]  modifier un test
                                \item[$\cdot$]  supprimer un test
                                \item[$\cdot$]  supprimer un commentaire
                        \end{itemize} & 
                        Authentification \\
                \hline
                        Superadmin & 
                        \begin{itemize}
                                \item[$\cdot$]  tous les besoins du visiteur
                                \item[$\cdot$]  tous les besoins d'un admin
                                \item[$\cdot$]  voir les logs
                                \item[$\cdot$]  ajouter un administrateur
                                \item[$\cdot$]  modifier un administrateur
                                \item[$\cdot$]  activer un administrateur
                                \item[$\cdot$]  désactiver un administrateur
                        \end{itemize} & 
                        Authentification \\
                \hline 
        \end{tabular}
        \caption{Tableau des utilisateurs et de leurs besoins} \label{tab:sometab}
\end{table}
\par
                \lipsum[1]
        \section{Approche de travail}
        \subsection{Le génie logiciel}
        \textit{
                En 1995, une étude du Standish Group dressait un tableau accablant de la 
                conduite des projets informatiques. Reposant sur un échantillon 
                représentatif de 365 entreprises, totalisant 8 380 applications, 
                cette étude établissait que \cite{audibert2009uml} :
                \begin{itemize}
                        \item 16,2\% seulement des projets étaient conformes 
                        aux prévisions initiales,
                        \item 52,7\% avaient subi des dépassements en coût et délai d'un facteur 2 à 3 
                        avec diminution du nombre des fonctions offertes,
                        \item 31,1\% ont été purement abandonnés durant leur développement.
                \end{itemize}
        }
        \paragraph{}
        GéoTechMap doit faire partie de ces 16,2 \%. Pour ce faire, il ne faut surtout pas
        négliger l'importance du génie logiciel.
        \par
        Le génie logiciel est un domaine de de recherche qui a pour objectif
        d'optimiser le coût de développement d'un logiciel. De ce fait, notre
        travail en tant qu'ingénieurs est de nous occuper de l'architecture
        du logiciel, en l'occurence ses composants ainsi que ses mécanismes.
        La conception passe par plusieurs phases. Ainsi, on établie une approche
        de travail qui permettera de répondre aux besoins grandissants du système 
        que l'on va concevoir.
        \par
        À la suite de l'évaluation et de la documentation des besoins spécifiques
        de l'URGéo, des utilisateurs et des spécifications logiques et matérielles
        relatif au système, un plan a été dréssé:
        \begin{itemize}
                \item l'analyse des besoins,
                \item l'élaboration des spécifications,
                \item la conceptualisation,
                \item le développement,
                \item la phase de test,
                \item déploiement,
                \item vulgarisation,
                \item la maitenance
        \end{itemize}
        \paragraph{Un système de qualité}
        \paragraph{}
        Il s'agira d'offir un logiciel de qualité qui s'appuera sur différents facteurs.
        GéoTechMap rempli exactement les fonction escomptés spécifieés dans le 
        cahier des charges. La validité du système ne pourra être mis en doute.
        De plus, ce sera un une application fiable et robuste, pouvant facilement
        être combiné avec d'autres logiciels (choix de développement API). Dans le 
        Tableau \ref*{tab:facteurs}, figurent les differents facteurs sur lesquels reposent la qualité
        de GéoTechMap.

\par    
\begin{table}
        \centering
        \begin{tabular}{|p{0.20\linewidth}|p{0.80\linewidth}|}
        \hline
                \textbf{Facteurs} & \textbf{Détails} \\
                \hline
                Facilité d'emploi &
                facilité d'apprentissage, d'utilisation, de préparation des données, 
                d'interprétation des erreurs et de rattrapage en cas d'erreur d'utilisation.
                         \\
                \hline
                Validité&
                aptitude d'un produit logiciel à remplir exactement ses fonctions, 
                définies par le cahier des charges et les spécifications
                    \\
                \hline
                Fiabilité &
                aptitude d'un produit logiciel à fonctionner dans des conditions anormales.
                    \\
                \hline
                Réutilisabilité&
                aptitude d'un logiciel à être réutilisé, en tout ou en partie, dans de nouvelles applications.
                    \\
                \hline
                Compatibilité&
                facilité avec laquelle un logiciel peut être combiné avec d'autres logiciels.
                        \\
                \hline
                Efficacité&
                Utilisation optimale des ressources matérielles.
                        \\
                \hline
                Portabilité&
                facilité avec laquelle un logiciel peut être transféré sous différents environnements matériels et logiciels.
                        \\
                \hline
                Vérifiabilité&
                facilité de préparation des procédures de test.
                        \\
                \hline 
                Intégrité&
                aptitude d'un logiciel à protéger son code et ses données contre des accès non autorisés.
                        \\

                \hline 
        \end{tabular}
        \caption{Quelques facteurs sur lesquels reposent la qualité
        de GéoTechMap \cite{audibert2009uml}} \label{tab:facteurs}
\end{table}
\par

        \subsection{Le cycle de vie du logiciel}
        Ce cycle désigne les principales étapes de développement du logiciel.
        Le but de cette sécantation est de permettre la vérification du processus de déveeloppement.
        Il comprend le plus souvent les étapes suivantes :
        \begin{itemize}
                \item L'analyse des besoins et la faisabilité du projet
                \item La conception
                \item Le codage
                \item Les tests
                \item La documentation
                \item La mise en production
                \item La maintenance
        \end{itemize}
        Le cycle de vie peut être modélisé de plusieurs manières (Figure \ref{fig:methv} ).
        Nous utiliserons ici le modèle du cycle en V car demeure actuellement le cycle de vie le 
        plus connu et certainement le plus utilisé. 
        Il s'agit d'un modèle en cascade dans lequel le 
        développement des tests et du logiciel sont effectués 
        de manière synchrone \cite{audibert2009uml}.
        \begin{figure}[ht!]
                \centering
                \begin{subfigure}{.45\linewidth}
                    \includegraphics[scale=0.35]{images/Analyse_des_besoins/methcasc.png}
                    \caption{Modèle du cycle de vie en cascade}
                \end{subfigure}
                \hskip2em
                \begin{subfigure}{.45\linewidth}
                    \includegraphics[scale=0.35]{images/Analyse_des_besoins/methv.png}
                    \caption{Modèle du cycle de vie en V}
                \end{subfigure}
                \label{fig:methv}
            \end{figure}
           
       
        \section{Méthodologie}
        Pour la réalisation d'un système complexe, le traitement des problèmes doit se faire 
        de manière efficace. Les tâches lourdes serons divisées en de petites et assignées 
        à chaque membre de l'équipe  en fonction de ses aptitudes à les résoudre. En peu de mots,
        la méthodologie Agile et scrum est adoptée dans le cadre de ce projet.

        
        \paragraph{C’est quoi, la méthode Agile et Scrum ?}
        \paragraph{}
        \textit{Agile représente un ensemble de “méthodes et pratiques basées sur 
        les valeurs et les principes du Manifeste Agile”, qui repose entre autre sur 
        la collaboration, l’autonomie et des équipes pluri-disciplinaires}\cite{Littlefield2017}.
        \par
        La méthodologie Agile s'oppose généralement à la méthodologie traditionnelle waterfall (en cascade :
        \textit{dès qu'une étape du projet est terminée, l'équipe passe à l'étape suivante ; il n'y a pas (ou peu) 
        de retour en arrière}\cite{2017}). Elle se veut plus souple 
        et adaptée, et place les besoins du client au centre des priorités du projet.
        \par
        \textit{Scrum est un framework qui est utilisé pour implémenter la méthode 
        Agile de développement et de gestion de projet}\cite{Littlefield2017}.

        \paragraph{}
        La méthode agile de gestion de projet et le framework Scrum est basé sur une méthode 
        itérative de livrables du produit. Au lieu d’attendre que le projet soit 100\% finalisé 
        pour le livrer au client, vous délivrez des tronçons “utilisables” du projet au cours du 
        temps. Vous éviterez ainsi de gaspiller des efforts en cas de nécessité de changement ou 
        de problème de communication. Au-delà de l’importance des itérations et des améliorations 
        pour le produit, Scrum s’attache également à améliorer le processus à chaque nouveau cycle.
        \paragraph{}
        Un projet Scrum peut être agencé de différentes manières mais sont toujours présents:
        \begin{itemize}
                \item Product Owner : Il représente les intérêts du client et à ce titre, il a 
                l’autorité pour définir les fonctionnalités du produit final. Dans notre cas, il s'agit de l'URGéo.
                \item Sprint : Scrum utilise des sprints comme intervalles de temps pendant lesquels l’équipe 
                va compléter un certain nombre de tâches. Chaque sprint se termine avec une Rétrospective, qui réunit 
                toute l’équipe afin de partager les retours d’expérience et discuter des améliorations possibles du 
                prochain sprint.
        \end{itemize}

        \paragraph{Pourquoi Agile et Scrum ?}
        \begin{itemize}
                \item Scrum est la méthode agile la plus éprouvée et la plus documentée.
                \item Contrairement à la méthode traditionnelle waterfall, l'approche Agile offre une plus 
        grande flexibilité et une meilleure visibilité dans la gestion du projet. 
                \item L'avantage majeur de l'approche Agile est sa flexibilité. Les changements du client et les imprévus 
        sont pris en compte et l'équipe projet peut réagir rapidement.
                \item Le client dispose d'une meilleure visibilité sur l'avancement du projet et peut ainsi l'ajuster 
                en fonction de ses besoins. Le contrôle qualité est permanent. Quant à l'équipe projet, elle peut 
                réagir rapidement aux demandes du client.\footnote{Qu'est-ce que la méthodologie Agile ? \url{https://www.planzone.fr/blog/quest-ce-que-la-methodologie-agile}}
        \end{itemize}
        
        
        

        \section{Structure modulaire}
                \lipsum[1]
        \section{Préparation des documents}
                

                \par
                Étant donnée que la majorité des rapports et résultats des tests sont 
                disponibles sous forme papier, la première étape a consisté à scanner les documents.
                Pour ce faire nous adoptons une protocole: on assure la traçabilité de chaque 
                document les munissant d'une cartouche reprenant une série d'informations.
                Voici la liste des informations qui forme une cartouche:
                \begin{itemize}
                        \item \textbf{Lieu: }
                        Il s'agit de l'endroit où l'étude a été effectuée.
                        \item \textbf{Date: }
                        Il s'agit de la date à laquelle l'étude a été effectuée.
                        \item \textbf{Maître d'ouvrage: }
                        Il s'agit du client pour lequel l'étude est effectuée.
                        \item \textbf{Maître d'œuvre : }
                        Il s'agit de la personne ou l'entreprise chargée de l'étude.
                        \item \textbf{Réf Numéro Étude: }
                        Il s'agit d'un identifiant unique permettant de tracer une étude.
                \end{itemize}
                
\par    
\begin{table} 
        \centering
        \begin{tabular}{|p{0.30\linewidth}|p{0.40\linewidth}|}
                \hline
                \textbf{Lieu} & Delmas \\
                \hline
                \textbf{Date} & Janvier 2020 \\
                \hline
                \textbf{Maître d'ouvrage} & Faculté Des Sciences \\
                \hline
                \textbf{Maître d'œuvre} & URGéo \\
                \hline
                \textbf{Réf Numéro Étude} & 1234 \\
                \hline
        \end{tabular}
        \caption{Exemple de cartouche sur un document d'étude géotechnique} \label{tab:example_cartouche}
\end{table}
\par

        \section{Modélisation avec UML}
        \section{Diagramme des cas d'utilisation}
    \subsection{Cas général}
    \begin{figure}
        \centering
        \includegraphics[width=1\textwidth]{casGeneral.png}
        \caption{Diagramme des cas d'utilisation général}
    \end{figure}
    
        
 