\chapter{Contexte}
\paragraph{}
L’Unité de Recherche en Géotechnique (URGéo) dans le cadre de ses activités de recherches et de service à
la communauté dispose d’un ensemble de données géotechniques, géophysiques et géologiques. Le projet
consiste à collecter et organiser ces données existantes dans un Système d’Information Géographique (SIG).
L’objectif de ces services est donc de concevoir, développer, tester et installer un système de gestion
informatisée (base des données) à utiliser dans la gestion de l’archive technique de l’URGéo. Cette unité
dispose d'une base de données constituée d’une collection de documents au format PDF (Portable Document
Format). Le format PDF est un standard ouvert d'échange de documents électroniques géré par l’ISO
(International Organization for Standardization). Il souhaite cependant réaliser la migration de cette base vers
un système de gestion plus efficace.
    \section{Introduction}
        \subsection{Géneral}
        \lipsum[1]
        \subsection{Objectif du projet}
        \lipsum[1]
    \section{Réalité des études de sol en Haiti}
    \url{https://lnbtp.gouv.ht/publications.html}
    \lipsum[1]
        \subsection{Les donnees géotechniques en Haiti}
        \lipsum[1]
        \subsection{Les instances responsables de ces études}
        \paragraph{Généralités}
            pkp
        \paragraph{URGéo}
        L'Unité de Recherche en Géosciences a pour mission de mener des recherches dans les
        domaines des géosciences où elle a les capacités de pour le faire.
        Cela implique une bonne compréhension des différentes problématiques liés au sol et
        au sous-sol et la proposition de moyens de mitigations adaptées à la réalité haïtienne.
        \cite{mission_urgeo}
        \par
        Pour le moment, l’URGéo constitue l’une des rares unités de recherches dédiée aux
        géosciences dans le pays. 
        Ces chercheurs prennent part à de grandes réunions savantes et scientifiques en
        Amérique du Nord, en Europe et dans les Caraïbes.
        \cite{urgeo_nouvelliste}
        \paragraph{BME}
        Le Bureau des Mines et de l’Energie (BME) est un organisme autonome créé en 
        1986 fonctionnant sous  tutelle du Ministre des Travaux Publics Transports 
        et Communications (MTPTC). Sa mission principale est de promouvoir la recherche
        et l'exploitation des ressources minérales et énergétiques d'Haíti ainsi que les 
        techniques appropriées y relative
        \paragraph{SICOD}
        \paragraph{LNBTP}
        \paragraph{Insoflor}
        \paragraph{Géothechsol}
        \subsection{Leurs réalisations}
        \lipsum[1]
        \section{Les BDD géotechniques dans le monde}
        \subsection{Dans les Caraibes}
        \lipsum[1]
        \subsection{En Amerique}
        \lipsum[1]
        \subsection{En Europe}
        \lipsum[1]
        \subsection{En Asie}
        \lipsum[1]
        \subsection{En Afrique}
        \lipsum[1]
    \section{Comparaison entre outils déjà implémentés}
    \lipsum[1]
    \section{Gestion actuelle des données géothechniques en Haiti}
    \lipsum[1]
    % sur papier ? gratuit ? demarche ? centralise 
    \subsection{Apport de ce projet(centralisation des data)}
        \lipsum[1]
        %penser aux objectifs