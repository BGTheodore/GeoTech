\chapter{Contexte}
\paragraph{}
L’Unité de Recherche en Géotechnique (URGéo) dans le cadre de ses activités de recherches et de service à
la communauté dispose d’un ensemble de données géotechniques, géophysiques et géologiques. Le projet
consiste à collecter et organiser ces données existantes dans un Système d’Information Géographique (SIG).
L’objectif de ces services est donc de concevoir, développer, tester et installer un système de gestion
informatisée (base des données) à utiliser dans la gestion de l’archive technique de l’URGéo. Cette unité
dispose d'une base de données constituée d’une collection de documents au format PDF (Portable Document
Format). Le format PDF est un standard ouvert d'échange de documents électroniques géré par l’ISO
(International Organization for Standardization). Il souhaite cependant réaliser la migration de cette base vers
un système de gestion plus efficace.
    \section{Introduction}
        \subsection{Géneral}
        \lipsum[1]
        \subsection{Objectif du projet}
        \lipsum[1]
    \section{Réalité des études de sol en Haiti}
    \paragraph{}
    Lors d'un projet de Génie civil, quelle que soit son envergure, une firme se doit d'étudier 
    les caractéristiques liées au sol de fondation. De façon générale, ces études peuvent s'étaler 

    sur des durées assez courtes ou longues, dépendamment de l'ampleur du travail à réaliser.
    \par
    Après le séisme de 2010, Haïti a accordé encore plus d'importance aux études de sol. Ces données s'averent 
    incontournables dans toute construction civile. Des études sont réalisées lors de la construction de grandes infrastructures ou de routes.
    Par exemple: la construction d'un centre déprtemental d'approvisionnement de en intrants pour la direction sanitaire 
    départementale du sud'est. \cite{realisation_geotechsol} |
    


        \subsection{Les données géotechniques en Haiti}
        \paragraph{Il nous faut plus d'informations sur la technique de ces études et la disponibilité
        des données.}
        \lipsum[1]
        \subsection{Les instances responsables de ces études}
        \paragraph{}
        Vu la sensibilité de ces données, la réalisation de ces études ne peut être confiée n'importe qui. De ce 
        fait, des instances reconnues dans le domaine sont continuellement contactées 
        lorsqu'une firme (ou un particulier) se retrouve dans le besoin.
        \paragraph{URGéo}
        L'Unité de Recherche en Géosciences a pour mission de mener des recherches dans les
        domaines des géosciences où elle a les capacités de pour le faire.
        Cela implique une bonne compréhension des différentes problématiques liés au sol et
        au sous-sol et la proposition de moyens de mitigations adaptées à la réalité haïtienne.
        \cite{mission_urgeo}
        \par
        Pour le moment, l’URGéo constitue l’une des rares unités de recherches dédiée aux
        géosciences dans le pays. 
        Ces chercheurs prennent part à de grandes réunions savantes et scientifiques en
        Amérique du Nord, en Europe et dans les Caraïbes.
        \cite{urgeo_nouvelliste}
        \paragraph{BME}
        Le Bureau des Mines et de l’Energie (BME) est un organisme autonome créé en 
        1986 fonctionnant sous  tutelle du Ministre des Travaux Publics Transports 
        et Communications (MTPTC). Sa mission principale est de promouvoir la recherche
        et l'exploitation des ressources minérales et énergétiques d'Haíti ainsi que les 
        techniques appropriées y relative
        \paragraph{SICOD}
        La  Société d’Ingénierie Constructions et d’Orientations Diverses (SICOD),
         fondée en 2011, est une société haïtienne en noms collectifs   qui évolue dans les domaines d’ingénierie géotechnique et de constructions.
         Il s'adonnent aux prélèvement des données des essais de laboratoire, des interpretations systématiques et aux recommandations techniques. 
         Ils apportent leur supoort techniques aux maitres d'ouvrages dans la réalisation de leur chantier tout en observant les critères techniques de l'art.
        \paragraph{LNBTP}
        Le LNBTP est une institution publique à gestion autonome chargée du contrôle de la qualité des infrastructures en construction dans le pays. Il s'occupe 
        aussi des études géotechniques, des recherches appliquées sur les matériaux de construction et de la promotion des normes en matière de génie civil.
        \paragraph{Géothechsol}
        Géothechsol est un Bureau d’Etudes en Ingénierie Géotechnique et Environnemental ainsi qu’en formulation de béton et ses essais mécaniques et physiques,
         qui s’est fixé pour objectif de vous apporter une réponse sérieuse et de qualité, adaptée à vos besoins dans le respect de vos contraintes.
         Ce bureau axe ses traveaux sur les essais géotechniques et des sondages.

        \subsection{Leurs réalisations}
        En général ces entreprises s'impliquent dans la contruction et la recherche. 
        Leur travail consiste à effectuer une reconnaissance/étude géotechnique des sites et des échantillons  sont sélectionnés pour des analyses au
        laboratoire.
        \par
        Depuis plusieurs années ils se sont fait remarqué, notemmentent dans
         l'etude des sols avant la construction de grands batiments. Ils sont aussi impliqués dans la réalisation des ponts et des routes sur le territoire
          haitien. Cependant la concurence est rude car des firmes etranées sont parfois appelées. 
        \section{Les BDD géotechniques dans le monde}
        \subsection{Dans les Caraibes}
        \paragraph{Elaboration d'une Base de Données Géotechniques
        sur 1'Ile de Cayenne: }
        Elle a été élaborée dans le cadre d'une convention passée entre la
        Région de Guyane et le BRGM en 2001.
        \cite{Cayenne}
         L'objectif était de constituer une base de données renseignée regroupant tous les points (sondages, essais
        in situ ou en laboratoire) améliorant la connaissance des caractéristiques géomécaniques des
        formations d'une zone de projet. Cela permetra de mieux appréhender les types de problèmes
        spécifiques au site, et donc de mieux dimensionner les campagnes de reconnaissance
        géotechniques, aussi bien sur le plan technique que financier.
       \paragraph{ Principes de fonctionnement de l'application ACCESS: }
       Une base de données est un ensemble d'informations associées à un sujet particulier.
        Microsofi Access, permet de gérer toutes les informations en respectant les relations définies
        par le modèle conceptuel des données, à l'aide d'un fichier unique de base de données. Dans
        ce fichier, les données sont réparties entre plusieurs contenants appelés tables. [...]
        Pour permettre la consultation de ces données, BD-GTC contient des formulaires qui
        permettent de consulta, d'ajouter et de mettre à jour les données des tables. L'ensemble de ces
        formulaires constitue l'application BD - GTC.
        \cite{Cayenne}

        \subsection{En Amérique}
        3 sci2003-web,
        Russelletal,
        07-0057
        \subsection{En Europe}
        4 Subsurface data base in geoenvironmental engineering,

        \lipsum[1]
        \subsection{En Asie}
        STAL9781607500315-3360,
         Scot,
         EGU2016-2503Poster,
         5 db development in bangkok
    

        

        \subsection{En Afrique}
        mongereau1988,
    \section{Comparaison entre des outils déjà implémentés}
    \lipsum[1]
    \section{Gestion actuelle des données géothechniques en Haiti}
    \lipsum[1]
    % sur papier ? gratuit ? demarche ? centralise 
    \subsection{Apport de ce projet(centralisation des data)}
        \lipsum[1]
        %penser aux objectifs