\chapter*{Contexte}
\paragraph{}
L’Unité de Recherche en Géotechnique (URGéo) dans le cadre de ses activités de recherches et de service à
la communauté dispose d’un ensemble de données géotechniques, géophysiques et géologiques. Le projet
consiste à collecter et organiser ces données existantes dans un Système d’Information Géographique (SIG).
L’objectif de ces services est donc de concevoir, développer, tester et installer un système de gestion
informatisée (base des données) à utiliser dans la gestion de l’archive technique de l’URGéo. Cette unité
dispose d'une base de données constituée d’une collection de documents au format PDF (Portable Document
Format). Le format PDF est un standard ouvert d'échange de documents électroniques géré par l’ISO
(International Organization for Standardization). Il souhaite cependant réaliser la migration de cette base vers
un système de gestion plus efficace.
    \section{Introduction}
        \subsection{Géneral}
        \lipsum[1]
        \subsection{Objectif du projet}
        \lipsum[1]
    \section{Réalité des études de sol en Haiti}
    \url{https://lnbtp.gouv.ht/publications.html}
    \lipsum[1]
    \section{Les donnees géotechniques en Haiti}
    \lipsum[1]
    \section{Les BDD géotechniques dans le monde}
        \subsection{Dans les Caraibes}
        \lipsum[1]
        \subsection{En Amerique}
        \lipsum[1]
        \subsection{En Europe}
        \lipsum[1]
        \subsection{En Asie}
        \lipsum[1]
    \section{Comparaison avec des outils déjà implémentés}
    \lipsum[1]
    \section{Gestion actuelle des données géothechniques en Haiti}
    \lipsum[1]
    \subsection{Apport de ce projet(centralisation des data)}
        \lipsum[1]