\chapter{Contexte}
\section{Introduction}
    \subsection{Généralités}
        \subsubsection{Importance des données géotechiques en Haïti}
            \par
            Avant d’investir des millions de dollards et des centaines d’heures pour 
            construire un bâtiment, les proprietaires fonciers doivent savoir si le 
            plancher peut supporter le bâtiment en question.Un sous-sol mou et 
            rempli d'air peut conduire à un dépôt plus fort que souhaité, ce qui 
            conduit à des fissures prématurées dans tout le bâtiment.
            \par
            De ce fait, le plus sage est de recourir au préalable à des études de sol.   
            Malgré la valeur que peut coûter de telles études, que ce soit en termes
            économique et/ou temporel, les caractéristiques d’un sol restent une
            information essentielle à bien des égards. De ce fait, des études sont 
            réalisées lors de la construction de grandes infrastructures ou de routes. 
  
        \subsubsection{Gestion des données géotechiques en Haïti}
            \par
            Les outils papiers utilisés pour le moment sont très vulnérable à des
            catastrophes comme des incendies ou des tremblement de terre. D'autres
            part, lorsqu'ils sont numérisé, les fiches référencement, contenant les
            informations relatives au dossier, sont souvent stockés sur supports
            dur. La perte des documents de référence entraînerait un travail
            colossal pour le recouvrement des informations relatives à chaque 
            dossier.
            \par
            Diverses instances séparées détiennent des données recueillies au cours
            de leurs études respectives. En effet, la sensibilité et l’importance de
            ces dernières exigent l’existence de responsables dédiés à cette fin. 
            Ainsi, lorsqu’un particulier a besoin de faire des études de sols, il 
            fait appel à des instances clés capable de les prendre en charge. 
            Parmi celles accessibles dans le pays, les plus contactées restent :
            \begin{itemize}
                \item \textbf{URGéo}
                L'Unité de Recherche en Géosciences a pour mission de mener des
                recherches dans les domaines des géosciences où elle a les capacités
                pour le faire.Cela implique une bonne compréhension des différentes 
                problématiques liés au sol et au sous-sol et la proposition de moyens
                de mitigations adaptées à la réalité haïtienne.
                \cite{mission_urgeo}
                \\
                Pour le moment, l’URGéo constitue l’une des rares unités de recherches
                dédiée aux géosciences dans le pays. Ces chercheurs prennent part à de
                grandes réunions savantes et scientifiques en Amérique du Nord, en 
                Europe et dans les Caraïbes.
                \item \textbf{BME}
                Le Bureau des Mines et de l’Energie (BME) est un organisme autonome créé en 
                1986 fonctionnant sous la tutelle du Ministre des Travaux Publics, Transports 
                et Communications (MTPTC). Sa mission principale est de promouvoir la recherche
                et l'exploitation des ressources minérales et énergétiques d'Haíti ainsi que les 
                techniques appropriées y relative.
                \item \textbf{SICOD}
                La  Société d’Ingénierie Constructions et d’Orientations Diverses (SICOD),
                fondée en 2011, est une société haïtienne en noms collectifs qui évolue dans 
                les domaines d’ingénierie géotechnique et de constructions.
                Il s'adonnent aux prélèvements des données des essais de laboratoire, des 
                interprétations systématiques et aux recommandations techniques. 
                Ils apportent leur support technique aux maîtres d'ouvrages dans la réalisation 
                de leur chantier tout en observant les critères techniques de l'art.
                \item \textbf{LNBTP}
                Le LNBTP est une institution publique à gestion autonome chargée du contrôle de
                la qualité des infrastructures en construction dans le pays. Il s'occupe 
                aussi des études géotechniques, des recherches appliquées sur les matériaux de 
                construction et de la promotion des normes en matière de génie civil.
                \item \textbf{Géothechsol}
                Géothechsol est un Bureau d’Etudes en Ingénierie Géotechnique et Environnemental
                ainsi qu’en formulation de béton et ses essais mécaniques et physiques, qui s’est
                fixé pour objectif de vous apporter une réponse sérieuse et de qualité, adaptée à 
                vos besoins dans le respect de vos contraintes. Ce bureau axe ses travaux sur les
                 essais géotechniques et des sondages.
            \end{itemize}      

    \subsection{Problématique}
    \textbf{Comment arriver à gerer de façon optimale les données géotechiques en Haïti et 
    mutualiser les données sur le sous-sol accumulées par différents organismes ?}
    \subsection{Annonce du plan de travail}

\section{Etude de l'existant}
    \subsection{Les BDD géotechniques dans le monde}
    \subsection{Apport de ce projet}
    \textit{Étant donné que cet outil n'existe pas 
    en Haïti, l'ampleur de ce projet fait donc surface.
    D'où l'implémentation}

\section{PLAN}
    \subsection{Implémentation d'une BDD géotechniques}
        \subsubsection{Numérisation des données}
        \subsubsection{Intégration de ces données dans une BDD}
    \subsection{Utilisation d'un GIS}
        \subsubsection{Connection de la carte et des infos de la BDD}
        \subsubsection{Utilisation de fonds de carte}
    \subsection{Création d'un web map} 
        \subsubsection{Implémentation d'un UI}
        \subsubsection{Publication de l'interface}

\section{Perspective de réalisation}