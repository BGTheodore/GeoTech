\chapter{Contexte}
\section{Introduction}
    \subsection{Généralités}
        \subsubsection{Importance des données géotechiques en Haïti}
          \par
Avant d’investir des millions de dollars et des centaines d’heures pour 
construire un bâtiment, les propriétaires fonciers doivent savoir si le 
plancher peut supporter le bâtiment en question. Un sous-sol mou et 
rempli d'air peut conduire à un dépôt plus fort que souhaité, ce qui 
conduit à des fissures prématurées dans tout le bâtiment.
\par
De ce fait, le plus sage est de recourir au préalable à des études de sol.   
Malgré la valeur que peut coûter de telles études, que ce soit en termes
économique et/ou temporel, les caractéristiques d’un sol restent une
information essentielle à bien des égards. De ce fait, des études sont 
réalisées lors de la construction de grandes infrastructures ou de routes. 
\par 
La zone d'étude de ce projet correspond à Haïti. 
Cette île située dans les Caraïbes a une superficie de  \SI{27750}{\kilo\metre\squared}.
\begin{figure}
    \centering
    \includegraphics[width=1\textwidth]{map-haiti.png}
    \caption{Cartographie d'Haïti}
\end{figure}
        \subsubsection{Gestion des données géotechiques en Haïti}
            \par
Les outils papiers utilisés pour le moment sont très vulnérables à des
catastrophes comme des incendies ou des tremblements de terre. D'autres
part, lorsqu'ils sont numérisées, les fiches référencement, contenant les
informations relatives aux dossiers, sont souvent stockées sur supports
durs. La perte des documents de référence entraînerait un travail
colossal pour le recouvrement des informations relatives à chaque 
dossier.
\par
Diverses instances séparées détiennent les données recueillies au cours
de leurs études respectives. En effet, la sensibilité et l’importance de
ces dernières exigent l’existence de responsables dédiés à cette fin. 
Ainsi, lorsqu’un particulier a besoin de faire des études de sols, il 
fait appel à des instances clés capable de les prendre en charge. 
Parmi celles accessibles dans le pays, les plus contactées restent :
\begin{itemize}
    \item \textbf{URGéo}
    Unité de Recherche en Géosciences 
    \item \textbf{BME}
    Bureau des Mines et de l’Energie
    \item \textbf{SICOD}
    Société d’Ingénierie Constructions et d’Orientations Diverses
    \item \textbf{LNBTP}
    Laboratoire National Du Bâtiment et des Travaux Publics 
    \item \textbf{Géothechsol}
\end{itemize}   

\par
En général ces entreprises s'impliquent dans la construction et la recherche. 
Leur travail consiste à effectuer une reconnaissance/étude géotechnique des sites et
des échantillons  sont sélectionnés pour des analyses au laboratoire.
\par
Depuis plusieurs années ils se sont fait remarquer, notamment dans
l'étude des sols avant la construction de grands bâtiments. Ils sont aussi impliqués
dans la réalisation des ponts et des routes sur le territoire
haitiens. 

    \subsection{Problématique}
    \textbf{Comment arriver à gerer de façon optimale les données géo\-techiques en Haïti et 
    mutualiser les données sur le sous-sol accumulées par différents organismes ?}
    \subsection{Panorama du projet}
        Avant d'entrer d'emblée dans le vif du sujet, nous aborborderons 
d'abord l'état de l'art. Cette phase va nous permetre de capitaliser le 
savoir et des savoir-faire existants, et de ne pas refaire des expériences 
qui auraient déjà été faites et dont les conclusions ont déjà été validées 
par des pairs.
\par
Par la suite, on se penchera sur les différents élements de réponse que l'on 
pourrait apporter au problème confronté.
Enfin nous metterons l'emphase sur l'implémentation des diverses solutions 
que l'on propose.


\section{Étude de l'existant}
    \subsection{Les BDD géotechniques dans le monde}
    \subsection{Apport de ce projet}
    \textit{Étant donné que cet outil n'existe pas 
    en Haïti, l'ampleur de ce projet fait donc surface.
    D'où l'implémentation}

\section{Cheminement de la solution}
    \subsection{Implémentation d'une BDD géotechniques}
        \subsubsection{Numérisation des données}
        \subsubsection{Intégration de ces données dans une BDD}
    \subsection{Utilisation d'un GIS}
        \subsubsection{Connection de la carte et des infos de la BDD}
        \subsubsection{Utilisation de fonds de carte}
    \subsection{Création d'un web map} 
        \subsubsection{Implémentation d'un UI}
        \subsubsection{Publication de l'interface}
    \begin{figure}[t]
        \centering
        \includegraphics[width=1\textwidth]{images/evolution_projetGIS.png}
        \caption{Cheminement de la solution}
    \end{figure}

\section{Perspective de réalisation}