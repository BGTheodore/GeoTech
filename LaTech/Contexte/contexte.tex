\chapter{Contexte}
\section{Introduction}
    \paragraph{}
    Ce mémoire s'inscrit dans le cadre d'une collaboration entre
    l'URGéo et Kay Nou Tek. Ce projet est financé par l’Ambassade
    de Suisse en Haïti et a pour objectif d’explorer le potentiel 
    du numérique pour améliorer les pratiques de construction.
    Le but principal de ce travail est
    d'apporter une solution technologique  qui facilitera la 
    gestion des données géotechniques en Haïti.
    \paragraph{}
    Ce document comporte deux grandes parties: une première axée
    sur la théorie, se focalisant sur le contexte et les avantages
    de la réalisation d'un tel projet. La seconde partie sera plus pratique
    et combinera le travail de l'ingénieur logiciel et des programmeurs responsables
    du projet.
    \begin{figure}
        \centering
        \includegraphics[width=1\textwidth]{map-haiti.png}
        \caption{Cartographie d'Haïti}
    \end{figure}
    \subsection{Généralités}
        \subsubsection{La géotechique, pourquoi est-ce important ?}
          \par
Avant d’investir des millions de dollars et des centaines d’heures pour 
construire un bâtiment, les propriétaires fonciers doivent savoir si le 
plancher peut supporter le bâtiment en question. Un sous-sol mou et 
rempli d'air peut conduire à un dépôt plus fort que souhaité, ce qui 
conduit à des fissures prématurées dans tout le bâtiment.
\par
De ce fait, le plus sage est de recourir au préalable à des études de sol.   
Malgré la valeur que peut coûter de telles études, que ce soit en termes
économique et/ou temporel, les caractéristiques d’un sol restent une
information essentielle à bien des égards. De ce fait, des études sont 
réalisées lors de la construction de grandes infrastructures ou de routes. 
\par 
La zone d'étude de ce projet correspond à Haïti. 
Cette île située dans les Caraïbes a une superficie de  \SI{27750}{\kilo\metre\squared}.
\begin{figure}
    \centering
    \includegraphics[width=1\textwidth]{map-haiti.png}
    \caption{Cartographie d'Haïti}
\end{figure}
        \subsubsection{Gestion des données géotechiques en Haïti}
            \par
Les outils papiers utilisés pour le moment sont très vulnérables à des
catastrophes comme des incendies ou des tremblements de terre. D'autres
part, lorsqu'ils sont numérisées, les fiches référencement, contenant les
informations relatives aux dossiers, sont souvent stockées sur supports
durs. La perte des documents de référence entraînerait un travail
colossal pour le recouvrement des informations relatives à chaque 
dossier.
\par
Diverses instances séparées détiennent les données recueillies au cours
de leurs études respectives. En effet, la sensibilité et l’importance de
ces dernières exigent l’existence de responsables dédiés à cette fin. 
Ainsi, lorsqu’un particulier a besoin de faire des études de sols, il 
fait appel à des instances clés capable de les prendre en charge. 
Parmi celles accessibles dans le pays, les plus contactées restent :
\begin{itemize}
    \item \textbf{URGéo}
    Unité de Recherche en Géosciences 
    \item \textbf{BME}
    Bureau des Mines et de l’Energie
    \item \textbf{SICOD}
    Société d’Ingénierie Constructions et d’Orientations Diverses
    \item \textbf{LNBTP}
    Laboratoire National Du Bâtiment et des Travaux Publics 
    \item \textbf{Géothechsol}
\end{itemize}   

\par
En général ces entreprises s'impliquent dans la construction et la recherche. 
Leur travail consiste à effectuer une reconnaissance/étude géotechnique des sites et
des échantillons  sont sélectionnés pour des analyses au laboratoire.
\par
Depuis plusieurs années ils se sont fait remarquer, notamment dans
l'étude des sols avant la construction de grands bâtiments. Ils sont aussi impliqués
dans la réalisation des ponts et des routes sur le territoire
haitiens. 

    \subsection{Problématique}
    \textbf{Comment arriver à créer une base de données permettant de 
    présenter et référencer l'ensemble des données géotechniques dans un Systéme
    d’Information Géographique (SIG) ?}

    \subsection{Panorama du projet}
        Avant d'entrer d'emblée dans le vif du sujet, nous aborborderons 
d'abord l'état de l'art. Cette phase va nous permetre de capitaliser le 
savoir et des savoir-faire existants, et de ne pas refaire des expériences 
qui auraient déjà été faites et dont les conclusions ont déjà été validées 
par des pairs.
\par
Par la suite, on se penchera sur les différents élements de réponse que l'on 
pourrait apporter au problème confronté.
Enfin nous metterons l'emphase sur l'implémentation des diverses solutions 
que l'on propose.


\section{Étude de l'existant}
    \subsection{Les BDD géotechniques dans le monde}
        \paragraph{}
Un système de gestion des informations géotechiques s'avère incontournable
dans un environnement de géoscience. Beaucoup d'universités et d'entreprises 
privées ainsi que l'état dans certains pays à travers le monde se sont déjà 
penchés sur la question. 
\paragraph{}
Les résutats divergent sur quelques détails à propos des technologies utilisées mais 
l'objectif est généralement le même: 
constituer une base de données renseignée regroupant tous les points (sondages, essais
in situ ou en laboratoire) améliorant la connaissance des caractéristiques géomécaniques des
formations d'une zone.
Par exemple, dans les Caraïbes, plus précisement sur l'Ile de Cayenne, cela a permis
\textit{ de mieux appréhender les types de problèmes
spécifiques au site, et donc de mieux dimensionner les campagnes de reconnaissance
géotechniques, aussi bien sur le plan technique que financier.}
\cite{Cayenne}
\par
L'une des faiblesses de certains projets a été l'utilisation des outils Microsoft
qui ne semblent pas assez 
adéquats. \textit{Ils sont trop génériques, ce qui empêche un stockage intelligent des données géotechniques.}
\cite{antoljak2012subsurface}
Ce problème est très courant dans de nombreuses entreprises qui s'attachaient au stylo et au papier
et qui n'ont trouvé comme meilleure solution que des feuilles de calcul.

%......................

\paragraph{}D'autres se basent de pŕeférence sur la \textit{Conception d'une architecture d'information 
géotechnique à l'aide de services Web.}
\cite{zimmermann2003design}
Cette architecture d'information a été implémentée à Los Angeles afin de permettre les échanges 
d'informations géotechniques accessibles pour tous. Les avantages apportés par une telle 
application pourraient tant se sentir pour des études concernant les risques sismiques que pour 
une meilleure approche lors des estimations effectuées par des compagnies d'assurance. 

\paragraph{}
Au Canada, plusieurs projets identiques ont vu le jour, notemment l'élabora\-tion d'une \textit{base 
de données géoscientifiques dans le but d’aider à la finalisation de la 
cartographie des dépôts en surface et en subsurface dans la région de la moraine d’Oak 
Ridges.}
\cite{russell1996regional}

%............................

\paragraph{CGD}
La base de données géotechniques de Canterbury (CGD) est une base de données en ligne qui a été développée pour
la reconstruction de Christchurch à la suite du tremblement de terre de Canterbury 2010-2011 (CES). Il
a été conçu comme un référentiel consultable pour le partage d'informations géotechniques existantes et nouvelles
ainsi que des applications géotechniques de soutien pour les autorisations de construction et de ressources. En mars
2015, la base de données contient plus de 18000 enregistrements d'essais de pénétration de cône, 4000 forages, 1000
piézomètres accompagnés de registres de surveillance des eaux souterraines, 6000 enregistrements de tests de laboratoire
plus d'autres données. 
\par
Ces données peuvent également être utilisées à des fins plus stratégiques telles que l'aide à la
relèvement en cas de futures catastrophes naturelles, accroissement de la résilience d'autres régions de la Nouvelle-Zélande,
modélisa\-tion des sinistres catastrophiques et information des processus réglementaires. La vaste base de données géotechniques
combinée à d'autres ensembles de données permet un examen et une modélisation approfondis du terrain
et la performance de l'infrastructure construite. Les leçons tirées de ces analyses peuvent être appliquées pour
améliorer la résilience et également utilisé pour éclairer les décisions de politique réglementaire dans d'autres domaines de
Zélande.
\par
Le CGD a été conçu comme un référentiel consultable pour les informations géotechniques existantes et nouvelles
ainsi que des applications géotechniques de soutien pour les autorisations de construction et de ressources. Tandis que
les données sont principalement utilisées pour la conception géotechnique de l'amélioration du sol, la fondation du bâtiment
réparations, fondations de nouveaux bâtiments et conception géotechnique pour les réparations d'infrastructures, il peut
également être utilisé à des fins plus stratégiques telles que l'aide à la récupération pour de futurs
catastrophes naturelles, augmentation de la résilience d'autres régions de la Nouvelle-Zélande.
\cite{scott2015benefits}


\begin{figure}[t]
\centering
\includegraphics[width=1\textwidth]{cgd.png}
\caption{Visualisation des résultats de la base de données de Canterbury.}
\end{figure}

%............................

\paragraph{}
L'Afrique ne fait pas exception à la liste des multiples pays ayant adopté l'idée
de concevoir une base do données géotechniques.
Par exemple, celle de la ville de Tunis (Tunisie) est orientée vers la cartographie géotechnique.
\par
Le modèle choisi a permis, après une analyse
pré1iminaire très importante, une description globale et
totale de toutes les données géologiques et géotechniques collectées sur le site de Tunis. I1
assure, de plus, une indépendance physique et logique, un partage des données (une même donnée accessible  
par plusieurs programmes), une non redondance des données, une non codification des
données géologiques, une grande facilité des relations
entre fichiers indépendants, une intégrité (validité)
totale des données. 
\textit{S'y ajoutent une souplesse remarquable d'interrogation de TUNIS-DATA-BANK
assurée par l'emploi d'un langage d'interrogation spécifique et l'utilisation des opérateurs et des connecteurs
logiques, une automatisation totale des tâches de la
phase de la manipulation de la base de données et une
sécurité totale des fichiers.}
\cite{tunis}

%..................

\paragraph{}
L’implémentation de tous ces SIG par des organismes internationaux résulte à des données considérées 
comme étant le système d’archivage officiel dans leur domaine de spécialité.
Le rythme de migration de ces données dans le SIG Web connait une croissance exponentielle. 
\par
Avec son mouvement vers le cloud et sur le Web, son intégration à l'information 
en temps réel via l'Internet des objets, le SIG est devenu une plateforme 
pertinente pour presque toutes les activités humaines - un système nerveux de 
la planète. Alors que notre pays est confronté au problèmes de gestion et de vulgarisation 
des données géotechiques, les SIG joueront un rôle de plus 
en plus important et 
fourniront un moyen de communiquer des solutions en utilisant le langage commun de 
la cartographie.
    \subsection{Avantages d'un Système de gestion des Informations Géotechniques}
        \paragraph{}
% Elle
% a été conçu comme un référentiel consultable pour le partage d'informations géotechniques existantes et nouvelles
% ainsi que des applications géotechniques de soutien pour les autorisations de construction et de ressources.
\textit{Un avantage d'un système de gestion de données géotechniques est 
la facilité avec laquelle les données 
peuvent être visualisées, filtrées et manipulées.
De plus, grâce aux règles métier, à la validation des données et aux 
processus de contrôle de qualité approfondis, le risque de trouver
des informations inexactes sont considérablement réduites. Une base de 
données correctement conçue ne nécessitera que les entrées de données
une fois, éliminant le besoin de ré-entrée et de reformatage. Une 
étude réalisée par Goldin et al.,
(2008) a montré qu'en moyenne 1,24\% des entrées de données dans Excel 
sont saisies de manière incorrecte; l'erreur alors
composés chaque fois que les données sont réintroduites. La mise en place 
«à entrée unique» d’une base de données bien conçue réduit les erreurs de 
transcription humaine, source majeure d’inexactitude pour les entreprises
traitant de grandes quantités de données géotechniques.}
\cite{keen2015development}
\paragraph{}
Étant donné que cet outil n'existe pas 
en Haïti, l'ampleur de ce projet fait donc surface.
D'où l'implémentation qui suit.

% \par
% Ces données peuvent également être utilisées à des fins plus stratégiques telles que l'aide à la
% relèvement en cas de futures catastrophes naturelles, accroissement de la résilience d'autres régions de la Nouvelle-Zélande,
% modélisa\-tion des sinistres catastrophiques et information des processus réglementaires. La vaste base de données géotechniques
% combinée à d'autres ensembles de données permet un examen et une modélisation approfondis du terrain
% et la performance de l'infrastructure construite. Les leçons tirées de ces analyses peuvent être appliquées pour
% améliorer la résilience et également utilisé pour éclairer les décisions de politique réglementaire dans d'autres domaines de
% Zélande.

\section{Cheminement de la solution}
    \subsection{Implémentation d'une BDD géotechniques}
        \subsubsection{Numérisation des données}
            \par
Au cours de la première étape, des données seront recueillies à 
travers diverses instances partenaires, principalement l'URGéo. 
Enregistrées sous divers formats(papiers, CSV, PDF entre autres), 
ces informations seront par la suite normalisées puis numérisées. En 
effet, une structure uniforme devra être imposée afin de satisfaire la 
compréhension de tout particulier. Par exemple, MPOKO JWENN...
        \subsubsection{Intégration de ces données dans une BDD}
            \par
Évidemment, une simple numérisation ne changerait point grand chose 
si les données restent stockées sur des disques à l'ancienne. Ainsi, 
la normalisation ayant apporté un standard au sein des informations 
enregistrées, ces dernières pourront parfaitement être intégrées dans 
une base de données créée à cette fin. Une fois implémentées, cette 
base pourra héberger toutes les informations géotechniques relatives 
à une analyse effectuée par l'une des instances concernées. Plus 
explicitement, l'URGéo pourra enregistrer les résultats obtenus lors 
d'un forage, en alimentant la BDD tout en respectant les critères de 
standardisations.
\paragraph{}
Bien qu'efficace, cette BDD géotechnique reste un concept assez 
abstrait pour un concerné direct qui ne verra aucune différence 
entre ce nouveau format et les fichiers auxquels il était 
précédemment habitué.
    \subsection{Utilisation d'un SIG}
        \subsubsection{Connection de la carte d'Haïti et de la BDD}
            \par
Comme réalisé à dans différents pays à travers le monde, la prochaine 
étape consistera à uiliser un Systeme d'Information Géographique (SIG) capable de faciliter l'interprétation 
scientifique de ces données. 
Les SIG permettent aux utilisateurs de créer leurs propres couches de cartes 
afin de résoudre des problèmes concrets. Les SIG ont également évolué ces dernières années pour 
devenir un moyen de partage de données et de collaboration, inspirant une 
vision qui devient aujourd’hui une réalité - une base de données qui 
couvre pratiquement tous les sujets; dans notre cas ce sera la géotechnique. 
Une fois le GIS lié à la base, tout intéressé pourra accéder aux informations 
enregistrées dans un format plus conventionnel. Cela facilitera la visualisation des données. 
Dans le cadre de ce projet, il 
pourra trouver les résultats des tests effectuées au niveau d'une zone précise.
        \subsubsection{Utilisation de fonds de carte}
            \par
Une fois les informations accessibles, l'interprétation devient 
plus évidente; ce qui peut, pourtant, s'avérer insufisant. Par 
ailleurs, des images relatives au contexte recherché par le scientifique 
resteront inévitables. De ce fait, différents fonds de carte seront mis 
à la disposition de ce dernier, facilitant sa manipulation des données. 
L'ingénieur civil voulant faire des études en hydraulique pourra ainsi 
interprêter les données relatives à son domaine en sélectionnant son 
contexte personnel.
\paragraph{}
Désormais, tout particulier pourra accéder aux données de la BDD 
géotechniques en se référant à son domaine d'ètude. Néanmoins, tout 
autre action de sa part demandera l'intervention d'un expert en base 
de données.
    \subsection{Visualisation des données} 
        \subsubsection{Implémentation d'un UI intégrant un webmap}
            \par
Finalement, la dernière étape consistera à mettre à la disposition 
de nos utilisateurs finaux un interface adéquat et facilement 
accessible, les permettant ainsi d'interagir avec la BDD. Grâce 
à cela, un administrateur pourra directement ajouter, afficher, modifier ou supprimer 
des informations sans avoir à contacter un expert en informatique.
Quant aux simple visiteur, ils auront la possibilité de visualiter les données sur une carte. 
Ces données vont permettre aux utilisateurs(ingénieurs, étudiants, etc) de prendre des décisions, 
d’analyser des situations précises, ou encore de donner des alertes par rapport à des évènements précis.
\par
En effet, l'autonomie de tous les utilisateurs sans formation préliminaire 
traduira la performance de l'application. L'expérience utilisateur n'est pas anodin dans 
le developpement d'un tel systeme.
        \subsubsection{Publication de l'interface}
            \par
Quelle serait l'utilité d'une application de cette envergure 
si sa portabilité n'était pas prise en compte ? - Aucune. Par 
conséquent, son déploiement dans le cloud relèvera d'un processus incontournable 
afin de la mettre à la disposition de tous les intéressés. Désormais, 
n'importe qui aura la possibilité d'accéder au portail web sans 
installation préalable. Néanmoins, pour une question de sécurité, 
certaines fomctionnalités exigeront à l'utilisateur/administrateur une authentification.
    \begin{figure}[t]
        \centering
        \includegraphics[width=1\textwidth]{images/evolution_projetGIS.png}
        \caption{Cheminement de la solution}
    \end{figure}

\section{Perspective de réalisation}
    \par
Étant plus que pragmatiques, nous ne limeterons pas à proposer 
uniquement une solution théorique. On va mettre la main dans la patte
et donner des résultats palpables et fonctionnels.
\par
Pour ce faire, nous déffissons un cheminement, un ensemble d'étapes à 
respecter pour aboutir à un résultat optimal au moindre coût.
Ce cursus comprend cinq grandes étapes: 

\begin{itemize}
    \item \textbf{L'initialisation du projet: }
    Cette étape marque le début de notre long parcours et aura comme principaux
    objets la prise de connaissance du problème (dans le CDC) et l'identification des voeux
    l'URGéo.
    \item \textbf{Planification: }
    Tout grand projet digne de ce nom doit être planifié. C'est au cours de cette étape
    que létat de l'art sera traité pour prendre connaissance de l'existant et s'inspirer des travaux
    similaires déjà réalisés. Puis viens la phase de l'analyse, de la évaluation des coûts du projet, 
    du choix  de l'architechture, des modèles,
    ainsi des techechnologies et des méthodes que l'on aura à utiliser.
    \item \textbf{Execution: }
    L'essence de cette étape se trouve dans la réalisation même du projet que ce soit en matiìere de base de 
    données ou de programmation.
    \item \textbf{Monitoring et controle: }
    Ici il s'agit d'effectuer des tests sur la qualité du produit final et et de vérifier si on a atteint le 
    résultat escompté. Notons que cette partie pourra se faire en parallèle avec l'éxecution en faisant de 
    l'intégration continue.
    \item \textbf{Fermeture: }
    Enfin, on aboutit à la clôture du projet apres déploiement et une potentielle période de maintenance.


\end{itemize}  