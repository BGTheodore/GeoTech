\subsection{Sécurité de l'application web}
    La sécurité des applications Web est un élément central de 
    toute entreprise basée sur le Web. La nature globale d'Internet 
    expose les propriétés Web à des attaques à partir de différents 
    endroits et à différents niveaux d'échelle et de complexité. 
    La sécurité des applications Web traite spécifiquement de la 
    sécurité des sites Web, des applications Web et des services 
    Web tels que les API.
    \par 
    Les attaques contre les applications Web vont de la manipulation 
    ciblée de bases de données à une perturbation du réseau à grande 
    échelle. Analysons ensemble quelques d’attaque les plus courantes
    et comment nous les contournons. N.B: Nous nous basons sur les publications
    de l'OWASP.
    OWASP signifie Open Web Application Security Project, une communauté 
    en ligne qui produit des articles, des méthodologies, de la documentation, 
    des outils et des technologies dans le domaine de la sécurité des applications 
    Web.
    
    
    \subsubsection{Injection}
        Une injection de code se produit lorsqu'un individu envoie des données 
        invalides à l'application Web avec l'intention de lui faire faire quelque 
        chose pour lequel l'application n'a pas été conçue. Plus précisement,
        l'injection SQL est un moyen courant par lequel les pirates et les 
        utilisateurs malveillants tentent de pirater des applications. Dans une 
        injection SQL (SQi), ils «injectent» des valeurs dans une requête de base de 
        données afin d’obtenir une visibilité sur la structure de la base de 
        données et éventuellement d’accéder aux données personnelles stockées 
        dans la base de données.
        \par 
        La cause principale d'une vulnérabilité d'injection de code est le manque 
        de validation et de désinfection des données utilisées par l'application Web.
        Pour faire face à cet type d'attage, nous utilisons une méthode un peu similaire
        à la solution native de Java "prepared statements" (Code \ref{lst:sqijava}). Il s'agit d'exécuter les 
        requête SQL dans Spring Boot en utilisant la classe NamedParameterJdbcTemplate (Code \ref{lst:sqisb}). 
        Cette méthode présente un avantage supplémentaire en fournissant plus de clarté en 
        remplaçant les points d'interrogation dans la requête par des noms significatifs.
        \begin{lstlisting}[caption={Éviter SQi en Java: Prepared statements},label={lst:sqijava},language=Java]
            public List<User> getUserByUserId(String userId)
            throws SQLException {    
            String sql = "select"
                + "first_name,last_name,username "
                + "from users where"
                + "userId = ?";
            
            Connection c = dataSource.getConnection();
            PreparedStatement p = c.prepareStatement(sql);
            p.setString(20, userId);
            ResultSet rs = p.executeQuery(sql)); 
        }
        \end{lstlisting}

        \begin{lstlisting}[caption={Éviter SQi avec Spring Boot: NamedParameterJdbcTemplate},label={lst:sqisb},language=Java]
            Map<String, Object> params = new HashMap<>();
            integer userId = 20;
            
                String sql = "select "
                + "first_name,last_name,username "
                + "from users where"
                + "userId = :userId";
            
            params.put("userId", userId);
            template.update(sql,params);
        \end{lstlisting}
        \paragraph{}
        De plus, pour éviter tout risque d'injection de code, nous utilisont un ORM (Object 
        Relational Mapping), qui encapsule tout interaction entre l'application et la
        base de données.

    \subsubsection{Broken Authentication}
        "Broken Authentication" est vulnérabilité peut permettre à un attaquant d'utiliser 
        des méthodes manuelles et / ou automatiques pour essayer de prendre le contrôle de 
        n'importe quel compte qu'il souhaite dans un système - ou pire encore - pour obtenir 
        un contrôle complet sur le système.
        \par 
        Broken Authentication fait généralement référence à des problèmes de logique 
        qui surviennent au niveau du mécanisme d’authentification de l’application - par exemple une 
        mauvaise gestion de session - lorsqu'un 
        acteur malveillant utilise des techniques de force brute pour deviner ou confirmer des 
        utilisateurs valides dans un système.
        \par 
        Pour minimiser les risques de Broken Authentication, nous ne laissons pas la page de connexion 
        des administrateurs accessible publiquement à tous les visiteurs du site Web. De plus nous prenons 
        un ensemble de mesures de précaution:
        \begin{itemize}
            \item 
        \end{itemize}