La hiérarchie visuelle est l'ordre dans lequel l'utilisateur traite les informations par niveau d'importance.
Dans la conception d'interface, ce concept est est très important.
\textit{La hiérarchie visuelle est la différence entre un site qui influence stratégiquement le flux et les 
décisions des utilisateurs, et un site qui «a l'air cool» \cite{cao20155}.}
\par 
Plusieurs facteurs ont été pris en compte pour créer une hiérarchie corecte:
\begin{itemize}
        \item La dimention des éléments; il faut attirer l'attention sur ce qui est le plus important
        tout en évitant que l'utilisateur soit distrait pas les élément les mois importants.
        \item La couleur est une ressource visuelle puissante, sa bonne utilisation permet de séparer 
        efficacement les éléments d'un écran pour les hiérarchiser ou les déprioriser. Dans la 
        conception d’interfaces, la couleur la plus forte est souvent celle de l’interaction, 
        car l’utilisateur a besoin d’agir ou de recevoir des commentaires du système.
        \item L'alignement: Tout élément qui se sépare de l'alignement des autres attirera l'attention. 
        En effet, l'alignement crée de l'ordre entre les éléments, tout changement de cette règle sera 
        considéré de manière spéciale pour la vue humaine et se démarquera.
        \item La répétition: Dans la conception d'interface, la répétition crée un sentiment 
        d'unité et de cohérence tout au long de l'expérience.
        \item Simplicité: Il est important de garder un plafond sur le nombre de façons dont l'accent est
         mis dans une conception et d'être cohérent sur la façon dont ils sont utilisés. 
         Cela réduira au minimum les distractions.
        \item ...\cite{design}

\end{itemize}
On le voit, la hiérarchie est un élément important de tout projet de conception de site Web. 
Elle détermine le flux et le succès de chaque page de notre site Web.


