La hiérarchie visuelle est l'ordre dans lequel l'utilisateur traite les informations par niveau d'importance.
Dans la conception d'interface, ce concept est est très important.
\textit{La hiérarchie visuelle est la différence entre un site qui influence stratégiquement le flux et les 
décisions des utilisateurs, et un site qui «a l'air cool» \cite{cao20155}.}
\par 
Plusieurs facteurs ont été pris en compte pour créer une hiérarchie corecte:
\begin{itemize}
    \item La dimention des éléments; il faut attirer l'attention sur ce qui est le plus important
    tout en évitant que l'utilisateur soit distrait pas les élément les mois importants.
    \item La couleur est une ressource visuelle puissante, sa bonne utilisation permet de séparer 
    efficacement les éléments d'un écran pour les hiérarchiser ou les déprioriser. Dans la 
    conception d’interfaces, la couleur la plus forte est souvent celle de l’interaction, 
    car l’utilisateur a besoin d’agir ou de recevoir des commentaires du système.
    \item 
\end{itemize}
\subsection{La hiérarchie du dashboard de l'administrateur}

\subsection{La hiérarchie du dashboard du visiteur}