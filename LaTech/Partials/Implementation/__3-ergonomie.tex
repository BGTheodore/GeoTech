Alain Wisner, un pionnier de l’ergonomie en France définit le terme ainsi:
\textit{L’ensemble des connaissances scientifiques relatives à l’Homme nécessaires pour 
concevoir des outils, des machines et des dispositifs qui puissent être utilisés avec 
le maximum de confort, de sécurité et d’efficacité.}
\paragraph{}
A l’heure actuelle, cette discipline s’applique parfaitement dans le cadre de la 
création de sites web. En effet, lors de leur conception, il est primordial de 
faciliter l’interaction entre l’homme et l’interface autant pour le propriétaire 
du site qui va l’administrer que de ses usagers (les internautes) qui vont le 
consulter. La réussite d’un site internet passe donc aussi par l’orientation 
de sa conception en fonction des utilisateurs de  ce dernier \cite{ergonomie}.  
\paragraph{}
La réussite de notre application web est due aux trois principaux élèments de l'ergonomie 
web suivants : utilité, utilisabilité et satisfaction.
\paragraph{Utilité}
Un internaute peut facilement réussir à réaliser l’action pour laquelle il est venu dans le système.
\paragraph{Utilisabilité}
Un internaute peut effectuer ses actions rapidement et de manière intuitive avec le moins 
d’erreurs possible. 
\paragraph{Satisfaction}
Les internautes devraient donc être satisfait une fois leur action terminée et réalisée avec succès 
car nous tâchons de "joindre l’utile à l’agréable". 
\paragraph{}
Nous plaçons l’expérience de vos internautes au coeur de tout, et augmentons par conséquent nos chances 
qu’ils l’apprécient, y reviennent plus souvent et en parlent autour d’eux.

