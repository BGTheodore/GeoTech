\par
Évidemment, une simple numérisation ne changerait point grand chose 
si les données restent stockées sur des disques comme à l'ancienne. Ainsi, 
la normalisation ayant apporté un standard et une uniformité au sein des informations 
enregistrées, ces dernières pourront parfaitement être intégrées dans 
une base de données créée à cette fin. Une fois implémentée, cette 
base pourra héberger toutes les informations géotechniques relatives 
à une analyse effectuée par l'une des instances concernées. Plus 
explicitement, l'URGéo pourra enregistrer les résultats obtenus lors 
d'un forage, en alimentant la BDD tout en respectant les critères de 
standardisation.
\paragraph{}
Bien qu'efficace, cette BDD géotechniques reste un concept assez 
abstrait pour un concerné direct qui ne verra aucune différence 
entre ce nouveau format et les fichiers auxquels il était 
précédemment habitué.