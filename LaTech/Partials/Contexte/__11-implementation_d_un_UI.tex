\par
Finalement, la dernière étape consistera à mettre à la disposition 
de nos utilisateurs finaux un interface adéquat et facilement 
accessible, leur permettant ainsi d'interagir avec la BDD. Grâce 
à cela, un administrateur pourra directement ajouter, afficher, modifier ou supprimer 
des informations sans avoir à contacter un expert en informatique.
Quant aux simples visiteurs, ils auront la possibilité de visualiter les données sur une carte. 
Ces données vont permettre aux utilisateurs(ingénieurs, étudiants, etc) de prendre des décisions, 
d’analyser des situations précises, ou encore de donner des alertes par rapport à des évènements précis.
\par
En effet, l'autonomie de tous les utilisateurs sans formation préliminaire 
traduira la performance de l'application. L'expérience utilisateur n'est pas anodin dans 
le développement d'un tel système.