
    \par
    Avant d’investir des millions de dollars et des centaines d’heures dans la 
    construction d'un bâtiment, les propriétaires fonciers doivent savoir si le 
    plancher peut supporter le bâtiment en question.
    Un sous-sol inadéquat à un type de construction peut conduire à des 
    dommages considérables. Au pire, cela peut causer des pertes en vies humaines 
    en cas d'effondrement.
    \par
    De ce fait, il est nécessaire de recourir au préalable à des études de sol.   
    Malgré la valeur que peut coûter de telles études, que ce soit en termes
    économique et/ou temporel, les caractéristiques d’un sol restent une
    information essentielle à bien des égards. Par conséquent, des études sont 
    réalisées lors de la construction de grandes infrastructures ou de routes. 
    
    \paragraph{}
    Vu l'importance capitale qu'ont les données géotechniques, l'URGéo a
    mis sur pied un projet visant à les exploiter au mieux.
    Ce mémoire s'inscrit dans le cadre d'une collaboration entre
    l'URGéo et Kay Nou Tek. Ce projet est financé par l’Ambassade
    de Suisse en Haïti et a pour objectif d’explorer le potentiel 
    du numérique pour améliorer les pratiques de construction.
    Le but principal de ce travail est
    d'apporter une solution technologique  qui facilitera la 
    gestion des données géotechniques en Haïti.
    \paragraph{}
    Ce document comporte deux grandes parties: une première axée
    sur la théorie, se focalisant sur le contexte et les avantages
    de la réalisation d'un tel projet. La seconde partie est plus pratique
    et combine le travail de l'ingénieur logiciel et des analystes programmeurs responsables
    du projet.
