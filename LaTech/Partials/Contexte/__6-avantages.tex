\paragraph{}
Les avantages apportés par le développement d'un systeme d'information géotechique
sont multiples.
L'un des plus important est la facilité avec laquelle les données 
peuvent être visualisées, filtrées et manipulées.
De plus, le risque de trouver  des informations inexactes est considérablement
réduite grâce aux protrocoles du métier, à la validation des données et aux processus
de contrôle approfondis.

\textit{ Une 
étude réalisée par Goldin et al.,
(2008) a montré qu'en moyenne 1,24\% des entrées de données dans Excel 
sont saisies de manière incorrecte; l'erreur alors
composés chaque fois que les données sont réintroduites. La mise en place 
«à entrée unique» d’une base de données bien conçue réduit les erreurs de 
transcription humaine, source majeure d’inexactitude pour les entreprises
traitant de grandes quantités de données géotechniques.}
\cite{keen}
\paragraph{}
Ce projet peut aussi être
conçu comme un référentiel consultable pour le partage d'informations 
géotechniques existantes et nouvelles.
Il sera un outil de soutien pour les autorisations de construction délivré par l'État haïtien.
\paragraph{}
\par
Ces données peuvent également être utilisées à des fins plus stratégiques telles que l'aide à la
relèvement en cas de futures catastrophes naturelles.
Elles peuvent être utiles pour  l'élaboration des processus réglementaires.
La vaste base de données géotechniques
combinée à d'autres ensembles de données permettra un examen et une modélisation approfondis du terrain
et la performance de l'infrastructure à construire. 
\par
Les leçons tirées de ces analyses peuvent être appliquées pour
améliorer la résilience et également utilisé pour éclairer les 
décisions de politique réglementaire dans d'autres domaines en Haïti.
\paragraph{}
En plus du partage des données, une base de données géotechniques
offre les avantages suivants:
\begin{itemize}
    \item Diminution des coûts d'explotation des données géotechiques
    \item Les professionels peuvent accéder plus facilement aux informations géotechniques
    fournis par d'autres, économisant certains frais d'enquête
    \item Des évaluations de haut niveau peuvent être effectuées pour des
    projets utilisant des informations de la zone environnante pour
    mieux renseigner le profil géotechnique avant de s'engager à
    études plus détaillées
    \item Les données d'une zone peuvent être
    accessible pour servir de référence pour le terrain  dans des contextes géologiques similaires
    \item Les fournisseurs d'infrastructure peuvent être mieux informés
    et mieux cibler les zones les plus vulnérables , et, suite à un événement, optimiser les réparations
    \item  Les données souterraines peuvent être fournies aux autorités réglementaires
    et les décideurs pour leur permettre de prendre
    les décisions d'aménagement du territoire et déterminer la pertinence
    des stratégies et solutions d'investissement
    \item  Les entrepreneurs spécialisés peuvent évaluer les opportunités
    investir dans des équipements spécialisés et améliorer les sols de construction
    \item Améliorer la modélisation de sinistres catastrophes pour les assurances et
    la gestion des dangers. Cela facilerait la réalisation de scénario approprié    
\end{itemize}
Cette liste n'est évidemment pas exhaustive. Vous trouverez d'autres avantages dans le Tableau \ref{tab:avantages}.

\paragraph{}
Cette base de données géotechiques permet aussi à la communauté
scientifique d'avoir accès à certaines données qui normalement seraient 
hosrs de porté cause des contraintes budgétaires. La recherche devient alors plus 
facile dans le domaine de la géotechique en Haïti.


\paragraph{}
Étant donné que cet outil n'existe pas encore
en Haïti, l'ampleur de ce projet fait donc surface.
D'où l'implémentation qui suit.


\par    
\begin{table}
        \centering
        \begin{tabular}{|p{0.10\linewidth}|p{0.80\linewidth}|}
        \hline
                \textbf{ } & \textbf{Avantages} \\
                \hline
                    1 &
                    Réduction du risque de trouver des informations inexactes
                    \\
                \hline 
                    2 &
                    Pas de duplication des données
                    \\
                \hline 
                    3 &
                    Création d'un référentiel consultable pour le partage d’informations géotechniques
                    \\
                \hline 
                    4 &
                    Diminution des coûts de maintenance des données géotechiques
                    \\
                \hline 
                    5 &
                    Disponibilité des données géotechiques élevée
                    \\
                \hline 
                    6 &
                    Sécurité des données géotechiques
                    \\
                \hline 
                    7 &
                    Faciliter les prises de décisions d’aménagement du territoire 
                    et déterminer la pertinence des stratégies et solutions d’investissement
                    \\
                \hline 
                    8 &
                    Améliorer la modélisation de sinistres catastrophes pour les assurances
                    \\
                \hline 
                    9 &
                    Gérer simplement des grands ensembles de données
                    \\
                \hline 
                    10 &
                    Acceder de manière simple et efficace aux données enregistrées
                    \\
                \hline 
                    11 &
                    Avoir une grande flexibilité
                    \\
                \hline 
                    12 &
                    Avoir l'intégrité et la cohérence des données
                    \\
                \hline 
                    13 &
                    Avoir le contrôle des accès pour les utilisateurs (sécurité et protection des données)
                    \\
                \hline 
                    14 &
                    L'indépendance entre données et traitements
                    \\
                \hline 
                    15 &
                    L'ordre dans le stockage de données
                    \\
                \hline 
                    16 &
                    L'utilisation simultanée des données par différents utilisateurs
                    \\
            \hline 
        \end{tabular}
        \caption{Liste de quelques avantages d'une base de données géotechniques} \label{tab:avantages}
\end{table}
\par
