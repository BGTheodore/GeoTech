\par
Comme réalisé à dans différents pays à travers le monde, la prochaine 
étape consistera à uiliser un Systeme d'Information Géographique (SIG) capable de faciliter l'interprétation 
scientifique de ces données. 
Les SIG permettent aux utilisateurs de créer leurs propres couches de cartes 
afin de résoudre des problèmes concrets. Les SIG ont également évolué ces dernières années pour 
devenir un moyen de partage de données et de collaboration, inspirant une 
vision qui devient aujourd’hui une réalité - une base de données qui 
couvre pratiquement tous les sujets; dans notre cas ce sera la géotechnique. 
Une fois le GIS lié à la base, tout intéressé pourra accéder aux informations 
enregistrées dans un format plus conventionnel. Cela facilitera la visualisation des données. 
Dans le cadre de ce projet, il 
pourra trouver les résultats des tests effectuées au niveau d'une zone précise.