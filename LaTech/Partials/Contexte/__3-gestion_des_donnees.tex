\par
Parallèlement, les outils papiers utilisés pour le moment sont très vulnérables à des
catastrophes comme des incendies ou des tremblements de terre. La perte 
des documents de référence entraînerait un travail
colossal pour le recouvrement des informations relatives à chaque 
dossier.
\par
Diverses instances détiennent les données recueillies au cours
de leurs études géotechniques respectives. 
Ainsi, lorsqu’un particulier a besoin de faire des études de sols, il 
fait appel à des instances clés capables de les prendre en charge. 
Parmi elles que nous avons pu contacter ou avec lesquels nous avons fait un partenariat, citons :
\begin{itemize}
    \item \textbf{URGéo}
    Unité de Recherche en Géosciences \cite{linkurgeo} 
    \item \textbf{BME}
    Bureau des Mines et de l’Energie \cite{linkbme} 
    \item \textbf{SICOD}
    Société d’Ingénierie Constructions et d’Orientations Diverses
    \item \textbf{LNBTP}
    Laboratoire National Du Bâtiment et des Travaux Publics \cite{linklnbtp} 
    \item \textbf{Géothechsol} \cite{linkgeotechsol} 
    \item \textbf{Insolflor} 
\end{itemize}   

\par
En général ces entreprises s'impliquent dans la construction et/ou la recherche. 
Leur travail consiste à effectuer une reconnaissance/étude géotechnique des sites .
Depuis plusieurs années ils se sont faits remarqués, notamment dans
l'étude des sols avant la construction de grands bâtiments. Ils sont aussi impliqués
dans la réalisation de ponts et de routes sur le territoire
haitien.
\paragraph{Microzonage}
\paragraph{}


\paragraph{}
Cependant un problème persiste: les données recueillies par ces instances
ne sont nullement en sécurité car elles sont stockées sur papier.
De plus, le minimum qui est numérisé n'est pas intégré dans un environnement 
dédié à cela.
L'analyse des données géotechniques sur toute l'étendue du territoire devient
encore plus difficile car aucune instance ne dispose de l'intégralité des tests effectués.
Cela implique une exploitation non optimale de ces données.
\paragraph{Les problèmes actuelles}
\paragraph{}
Le plus grand inconvénient dans la gestion actulles des données géotechiques
en Haïti est la sécurité de ces dernières. 
Cet aspect n'est pas anodin et doit être pris en compte dans la gestion d'un système
d'information.
Actuellement, les critères fondamentaux de la securité des données ne sont nullememt en vigueur en dans le cadre des 
des Systèmes d'Information Géotechniques.
\paragraph{Confidentialité : }
On n'a aucune garantie que seules les personnes autorisées 
aient accès aux données géotechiques. Le fait qu'elles soient
stockées que sur papier auguemente les risques qu'une personnes
n'ayant pas le droit d'accès puisse s'acquérir de ces données.
\paragraph{Intégrité : }
Actuellement personne ne peut
garantir que les données géotechniques que l'on a en notre possession 
sont bien celles que l’on croit. L'intégrité n'est pas assurée car le risque
pour que les données géotechniques soit altérées est trop grand.
\paragraph{Disponibilité :}
C'est l'un des plus grand inconvénient de la gestion actuelle des 
données géotechniques. Trouver une étude qui a été réalisée dans un endroit précis
ou à une date pricise n'est pas évidente. Cela coûte beaucoup de temps et de ressource pour effectuer
les recherche. Par conséquent, le facteur de disponibilité n'est pas 
au rendez-vous car le délai d'acces aux informations est trop long.

\paragraph{}
Divers autres problèmes peuvent être constatés (Tableau : \ref{tab:problemes} )  dans la gestion
des données géotechniques en Haïti. Notemment le fait que ces données
ne soient pas à l'abrit des catastrophes humaines (sabotage) et naturelles
(incendies, tremblement de terre, inondation, etc).


\par    
\begin{table}
        \centering
        \begin{tabular}{|p{0.10\linewidth}|p{0.80\linewidth}|}
        \hline
                \textbf{No} & \textbf{Problèmes} \\
                \hline
                    1&
                    Non disponibilité des données
                         \\
                \hline
                2&
                Données susceptibles aux catastrophes humaines et naturelles.
                    \\
                \hline
                3&
                Risques de récidive des réalisations de test
                    \\
                \hline
                4&
                Coûts liés à la gestion archaïque
                    \\
                \hline
                5&
                Risque élevée de la non intégrité
                        \\
                \hline
                6&
                Les données sont eparpillées
                        \\
                \hline
                7&
                Le traitement des données n'est pas évident
                        \\
                \hline
                8&
                Non exploitation des données par les spécialistes et universitaires
                        \\
                \hline 
                9&
                Non exploitation des données par l'état pour les prises de certaines décisions
                        \\
                \hline 
                10&
                Données non sécurisées
                        \\
                \hline 
        \end{tabular}
        \caption{Principaux problèmes liés à la gestion des données géotechniques en Haïti} \label{tab:problemes}
\end{table}
\par