\par
Étant plus que pragmatiques, nous ne nous limiterons pas à proposer 
uniquement une solution théorique. Nous mettrons la main à la pâte afin
de donner des résultats palpables et fonctionnels.
\par
Pour ce faire, nous définissons un cheminement, un ensemble d'étapes à 
respecter pour aboutir à un résultat optimal au moindre coût.
Ce cursus comprend cinq grandes étapes: 

\begin{itemize}
    \item \textbf{L'initialisation du projet: }
    Cette étape marque le début de notre long parcours et aura comme principaux
    objets la prise de connaissance du problème (dans le CDC) et l'identification des vœux de
    l'URGéo.
    \item \textbf{Planification: }
    Tout grand projet digne de ce nom doit être planifié. C'est au cours de cette étape
    que l'état de l'art sera traité pour prendre connaissance de l'existant et s'inspirer des travaux
    similaires déjà réalisés. Puis vient la phase de l'analyse, de l'évaluation des coûts du projet, 
    du choix  de l'architecture, des modèles,
    ainsi que des technologies et des méthodes que l'on aura à utiliser.
    \item \textbf{Exécution: }
    L'essence de cette étape se trouve dans la réalisation même du projet, que ce soit en matière de base de 
    données ou de programmation.
    \item \textbf{Monitoring et contrôle: }
    Ici, il s'agit d'effectuer des tests sur la qualité du produit final et de vérifier si on a atteint le 
    résultat escompté. Notons que cette partie pourra se faire en parallèle avec l'éxécution, en faisant de 
    l'intégration continue.
    \item \textbf{Fermeture: }
    Enfin, on aboutit à la clôture du projet après déploiement et à une potentielle période de maintenance.


\end{itemize}  