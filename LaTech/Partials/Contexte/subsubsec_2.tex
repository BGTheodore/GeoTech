\par
Les outils papiers utilisés pour le moment sont très vulnérables à des
catastrophes comme des incendies ou des tremblements de terre. D'autres
part, lorsqu'ils sont numérisées, les fiches référencement, contenant les
informations relatives aux dossiers, sont souvent stockées sur supports
durs. La perte des documents de référence entraînerait un travail
colossal pour le recouvrement des informations relatives à chaque 
dossier.
\par
Diverses instances séparées détiennent les données recueillies au cours
de leurs études respectives. En effet, la sensibilité et l’importance de
ces dernières exigent l’existence de responsables dédiés à cette fin. 
Ainsi, lorsqu’un particulier a besoin de faire des études de sols, il 
fait appel à des instances clés capable de les prendre en charge. 
Parmi celles accessibles dans le pays, les plus contactées restent :
\begin{itemize}
    \item \textbf{URGéo}
    Unité de Recherche en Géosciences 
    \item \textbf{BME}
    Bureau des Mines et de l’Energie
    \item \textbf{SICOD}
    Société d’Ingénierie Constructions et d’Orientations Diverses
    \item \textbf{LNBTP}
    Laboratoire National Du Bâtiment et des Travaux Publics 
    \item \textbf{Géothechsol}
\end{itemize}   

\par
En général ces entreprises s'impliquent dans la construction et la recherche. 
Leur travail consiste à effectuer une reconnaissance/étude géotechnique des sites et
des échantillons  sont sélectionnés pour des analyses au laboratoire.
\par
Depuis plusieurs années ils se sont fait remarquer, notamment dans
l'étude des sols avant la construction de grands bâtiments. Ils sont aussi impliqués
dans la réalisation des ponts et des routes sur le territoire
haitiens. 