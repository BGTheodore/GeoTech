\par
Les outils papiers utilisés pour le moment sont très vulnérables à des
catastrophes comme des incendies ou des tremblements de terre. La perte 
des documents de référence entraînerait un travail
colossal pour le recouvrement des informations relatives à chaque 
dossier.
\par
Diverses instances détiennent les données recueillies au cours
de leurs études géotechiques respectives. 
Ainsi, lorsqu’un particulier a besoin de faire des études de sols, il 
fait appel à des instances clés capables de les prendre en charge. 
Parmi celles accessibles dans le pays, les plus contactées restent :
\begin{itemize}
    \item \textbf{URGéo}
    Unité de Recherche en Géosciences 
    \item \textbf{BME}
    Bureau des Mines et de l’Energie
    \item \textbf{SICOD}
    Société d’Ingénierie Constructions et d’Orientations Diverses
    \item \textbf{LNBTP}
    Laboratoire National Du Bâtiment et des Travaux Publics 
    \item \textbf{Géothechsol}
\end{itemize}   

\par
En général ces entreprises s'impliquent dans la construction et la recherche. 
Leur travail consiste à effectuer une reconnaissance/étude géotechnique des sites .
Depuis plusieurs années ils se sont faits remarqués, notamment dans
l'étude des sols avant la construction de grands bâtiments. Ils sont aussi impliqués
dans la réalisation de ponts et de routes sur le territoire
haitien. 
\paragraph{}
Cepandant un problème persiste: les données recueillies par ces instances
ne sont nullememt en sécurité car elle sont stockées sur papier.
De plus, le minimum qui est numérisé n'est pas intégré dans un environnement 
dédié à cela.
L'analyse des données géotechiques sur toute l'étendue du territoire devient
encore plus difficile car aucune instance ne dispose de l'integralité des tests effectués.
 Cela implique une exploitation non optimale de ces données.
