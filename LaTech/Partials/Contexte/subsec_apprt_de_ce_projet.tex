\paragraph{}
Les avantages apportés par le développement d'un systeme d'information géotechique
sont multiples.
L'un des plus important est la facilité avec laquelle les données 
peuvent être visualisées, filtrées et manipulées.
De plus, le risque de trouver  des informations inexactes est considérablement
réduite grâce aux protrocoles du métier, à la validation des données et aux processus
de contrôle approfondis.

\textit{ Une 
étude réalisée par Goldin et al.,
(2008) a montré qu'en moyenne 1,24\% des entrées de données dans Excel 
sont saisies de manière incorrecte; l'erreur alors
composés chaque fois que les données sont réintroduites. La mise en place 
«à entrée unique» d’une base de données bien conçue réduit les erreurs de 
transcription humaine, source majeure d’inexactitude pour les entreprises
traitant de grandes quantités de données géotechniques.}
\cite{keen2015development}
\paragraph{}
Ce projet peut aussi être
conçu comme un référentiel consultable pour le partage d'informations 
géotechniques existantes et nouvelles.
Il sera un outil de soutien pour les autorisations de construction délivré par l'État haïtien.
\paragraph{}
\par
Ces données peuvent également être utilisées à des fins plus stratégiques telles que l'aide à la
relèvement en cas de futures catastrophes naturelles.
Elles peuvent être utiles pour  l'élaboration des processus réglementaires.
La vaste base de données géotechniques
combinée à d'autres ensembles de données permettra un examen et une modélisation approfondis du terrain
et la performance de l'infrastructure à construire. 
\par
Les leçons tirées de ces analyses peuvent être appliquées pour
améliorer la résilience et également utilisé pour éclairer les 
décisions de politique réglementaire dans d'autres domaines en Haïti.
\paragraph{}
Étant donné que cet outil n'existe pas 
en Haïti, l'ampleur de ce projet fait donc surface.
D'où l'implémentation qui suit.

