\paragraph{}
\textit{Un avantage d'un système de gestion de données géotechniques est 
la facilité avec laquelle les données 
peuvent être visualisées, filtrées et manipulées.
De plus, grâce aux règles métier, à la validation des données et aux 
processus de contrôle de qualité approfondis, le risque de trouver
des informations inexactes sont considérablement réduites. Une base de 
données correctement conçue ne nécessitera que les entrées de données
une fois, éliminant le besoin de ré-entrée et de reformatage. Une 
étude réalisée par Goldin et al.,
(2008) a montré qu'en moyenne 1,24\% des entrées de données dans Excel 
sont saisies de manière incorrecte; l'erreur alors
composés chaque fois que les données sont réintroduites. La mise en place 
«à entrée unique» d’une base de données bien conçue réduit les erreurs de 
transcription humaine, source majeure d’inexactitude pour les entreprises
traitant de grandes quantités de données géotechniques.}
\cite{keen2015development}
\paragraph{}
Étant donné que cet outil n'existe pas 
en Haïti, l'ampleur de ce projet fait donc surface.
D'où l'implémentation qui suit.