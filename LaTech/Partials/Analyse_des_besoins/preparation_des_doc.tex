

                \par
                Étant donnée que la majorité des rapports et résultats des tests sont 
                disponibles sous forme papier, la première étape a consisté à scanner les documents.
                Pour ce faire nous adoptons une protocole: on assure la traçabilité de chaque 
                document les munissant d'une cartouche reprenant une série d'informations.
                Voici la liste des informations qui forme une cartouche:
                \begin{itemize}
                        \item \textbf{Lieu: }
                        Il s'agit de l'endroit où l'étude a été effectuée.
                        \item \textbf{Date: }
                        Il s'agit de la date à laquelle l'étude a été effectuée.
                        \item \textbf{Maître d'ouvrage: }
                        Il s'agit du client pour lequel l'étude est effectuée.
                        \item \textbf{Maître d'œuvre : }
                        Il s'agit de la personne ou l'entreprise chargée de l'étude.
                        \item \textbf{Réf Numéro Étude: }
                        Il s'agit d'un identifiant unique permettant de tracer une étude.
                \end{itemize}
                
\par    
\begin{table} 
        \centering
        \begin{tabular}{|p{0.30\linewidth}|p{0.40\linewidth}|}
                \hline
                \textbf{Lieu} & Delmas \\
                \hline
                \textbf{Date} & Janvier 2020 \\
                \hline
                \textbf{Maître d'ouvrage} & Faculté Des Sciences \\
                \hline
                \textbf{Maître d'œuvre} & URGéo \\
                \hline
                \textbf{Réf Numéro Étude} & 1234 \\
                \hline
        \end{tabular}
        \caption{Exemple de cartouche sur un document d'étude géotechnique} \label{tab:example_cartouche}
\end{table}
\par
