\textit{
                En 1995, une étude du Standish Group dressait un tableau accablant de la 
                conduite des projets informatiques. Reposant sur un échantillon 
                représentatif de 365 entreprises, totalisant 8 380 applications, 
                cette étude établissait que \cite{audibert2009uml} :
                \begin{itemize}
                        \item 16,2\% seulement des projets étaient conformes 
                        aux prévisions initiales,
                        \item 52,7\% avaient subi des dépassements en coût et délai d'un facteur 2 à 3 
                        avec diminution du nombre des fonctions offertes,
                        \item 31,1\% ont été purement abandonnés durant leur développement.
                \end{itemize}
        }
        \paragraph{}
        GéoTechMap doit faire partie de ces 16,2 \%. Pour ce faire, il ne faut surtout pas
        négliger l'importance du génie logiciel.
        \par
        Le génie logiciel est un domaine de de recherche qui a pour objectif
        d'optimiser le coût de développement d'un logiciel. De ce fait, notre
        travail en tant qu'ingénieurs est de nous occuper de l'architecture
        du logiciel, en l'occurence ses composants ainsi que ses mécanismes.
        La conception passe par plusieurs phases. Ainsi, on établie une approche
        de travail qui permettera de répondre aux besoins grandissants du système 
        que l'on va concevoir.
        \par
        À la suite de l'évaluation et de la documentation des besoins spécifiques
        de l'URGéo, des utilisateurs et des spécifications logiques et matérielles
        relatif au système, un plan a été dréssé:
        \begin{itemize}
                \item l'analyse des besoins,
                \item l'élaboration des spécifications,
                \item la conceptualisation,
                \item le développement,
                \item la phase de test,
                \item déploiement,
                \item vulgarisation,
                \item la maitenance
        \end{itemize}
        \paragraph{Un système de qualité}
        \paragraph{}
        Il s'agira d'offir un logiciel de qualité qui s'appuera sur différents facteurs.
        GéoTechMap rempli exactement les fonction escomptés spécifieés dans le 
        cahier des charges. La validité du système ne pourra être mis en doute.
        De plus, ce sera un une application fiable et robuste, pouvant facilement
        être combiné avec d'autres logiciels (choix de développement API). Dans le 
        Tableau \ref*{tab:facteurs}, figurent les differents facteurs sur lesquels reposent la qualité
        de GéoTechMap.

\par    
\begin{table}
        \centering
        \begin{tabular}{|p{0.20\linewidth}|p{0.80\linewidth}|}
        \hline
                \textbf{Facteurs} & \textbf{Détails} \\
                \hline
                Facilité d'emploi &
                facilité d'apprentissage, d'utilisation, de préparation des données, 
                d'interprétation des erreurs et de rattrapage en cas d'erreur d'utilisation.
                         \\
                \hline
                Validité&
                aptitude d'un produit logiciel à remplir exactement ses fonctions, 
                définies par le cahier des charges et les spécifications
                    \\
                \hline
                Fiabilité &
                aptitude d'un produit logiciel à fonctionner dans des conditions anormales.
                    \\
                \hline
                Réutilisabilité&
                aptitude d'un logiciel à être réutilisé, en tout ou en partie, dans de nouvelles applications.
                    \\
                \hline
                Compatibilité&
                facilité avec laquelle un logiciel peut être combiné avec d'autres logiciels.
                        \\
                \hline
                Efficacité&
                Utilisation optimale des ressources matérielles.
                        \\
                \hline
                Portabilité&
                facilité avec laquelle un logiciel peut être transféré sous différents environnements matériels et logiciels.
                        \\
                \hline
                Vérifiabilité&
                facilité de préparation des procédures de test.
                        \\
                \hline 
                Intégrité&
                aptitude d'un logiciel à protéger son code et ses données contre des accès non autorisés.
                        \\

                \hline 
        \end{tabular}
        \caption{Quelques facteurs sur lesquels reposent la qualité
        de GéoTechMap \cite{audibert2009uml}} \label{tab:facteurs}
\end{table}
\par