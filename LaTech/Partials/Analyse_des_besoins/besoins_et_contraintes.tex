Nous débutons la conception de notre système en analysant la
situation pour prendre note des différentes contraintes, des risques
et tout autre élément pertinent dans le but de satisfaire l'intégralité
des besoins de l'URGéo.  Nous sommes déjà imbus du contexte de développement
du système, par conséquent, nous allons, dans cette partie, nous concentrer
sur les besoins et les contraintes de l'application.
\section{Besoins et contraintes}
        Il s'agit de la conception d'une base de données géotechniques et d'une
        application web permettant de visualiser cesdites données. Définissons
        d'abord tous les besoins des différents utilisateurs du système.
        \subsubsection{Identification des acteurs du système}
        Pour connaître les différents besoins des utilisateurs, nous devons
        avant tout relever la liste des différents utilisateurs eux-mêmes.
        Nombreux sont ceux qui auront à utiliser le système. Nous appellerons ces différents
        utilisateurs  les \textbf{acteurs} du système.
        \par
        L'application sera disponible pour tout le monde notemment les
        professionnels en géosciences, les ingénieurs, les étudiants, les banques, les 
        compagnies d'assurance, etc.
        Ces acteurs sont divisés en trois (3) catégories:
        \begin{itemize} 
                \item \textbf{les visiteurs: }
                Un visiteur est un utilisateur lambda qui se rend sur l'application pour rechercher et visualiser
                les données mises disponibles par l'URGéo et les instances associées. 
              
                \item \textbf{les administrateurs: }
                Un administrateur est un utilisateur avec plus de privilèges. Il a non seulement les droits d'un visiteur,
                mais aussi d'autres attributions. Il est obligatoire pour lui de s'authentifier pour pouvoir 
                effectuer certaines actions sur les systèmes. Seront administrateurs, toute personne désignée par l'URGéo
                ou les partenaires de l'URGéo. Le plus souvent, ce seront les stagiares responsables de l'entrée des données.

                \item \textbf{les superadmins: }
                Un super-administrateur est un super utilisateur. Il a non seulement les droits d'un visiteur,
                ceux de l'administrateur mais aussi d'autres attributions. Il est obligatoire pour lui de s'authentifier pour pouvoir 
                effectuer certaines actions sur les systèmes. Seront superadmins, toute personne désignée par l'URGéo.
            \end{itemize}   
        \subsubsection{Besoins des différents utilisateurs}
        Étant donné que l'on aura deux types d'utilisateurs avec des privilèges différents,
        le système doit impérativement comporter un mode de gestion des utilisateurs et des droits d'accès.
        \paragraph{Le visiteur}
        Le visiteur a à sa disposition une carte d'Haïti marquée aux différents endroits où des tests 
        géotechniques ont été réalisés.
        À n'importe quel moment, il peut décider d'effectuer une recherche par mot clé et s'attend
        à ce que le résultat de sa recherche s'affiche sur la carte. Il a aussi l'option de l'afficher sous la forme
        d'une liste, il a la possibilité de filtrer selon son choix. Cette dernière peut être téléchargée sous format CSV.
        En support aux informations spécifiques à un test se trouvant à un endroit bien précis sur la carte,
        le visiteur a aussi l'accès au résultat du test se trouvant dans un fichier PDF qu'il peut télécharger.
        \par
        Aussi, plusieurs fonds de carte seront disponibles permettant au visiteur d'adapter le résultat de ses recherches
        au contexte idéal (topographie, hydraulique,... )
        \par
        Le visiteur peut aussi décider de lire, de commenter ou de laisser un message (de manière anonyme ou pas) sur le forum dédié à l'application.
        \paragraph{L'administrateur}
        Avant tout, il peut réaliser toutes les actions d'un visiteur. De plus, après s'être authentifié au moyen de 
        son adresse électronique et de son mot de passe, il peut interagir directement avec la base de données. En cas 
        d'oubli de son mot de passe, le système lui envoie un lien de réinitialisation de mot de passe à son email.
        Pour jouer son rôle d'administrateur, il est redirigé vers \textit{l'interface de l'administrateur}. 
        Dans ce module, l'administrateur peut:
        \begin{itemize}
                \item \textbf{Ajouter un test: }
                Il s'agit de rentrer les informations relatives à un test pour l'ajouter dans la base de données.
                Ces informations sont de types différents (nom:texte, identifiant:entier, date du test:date, types
                de test:entier, date d'enregistrement:date, etc\footnote{Les différents champs et leur type seront 
                détaillés dans l'étude des diagrammes à la fin du chapitre} )
                \item \textbf{Modifier un test: }
                Si pour une raison ou pour une autre un test doit être modifié, l'administrateur est en
                mesure de le faire après s'être authentifié. Un message lui sera affiché à l'écran dépendemment 
                de la réussite ou de l'échec de son action.
                \item \textbf{Supprimer un test: }
                La suppression d'un test est aussi possible. Un message de confirmation précède la validation
                de l'exécution de cette action car elle est irréversible.
        \end{itemize}
        \par
        De plus, si l'URGéo juge que le commentaire d'un visiteur doit être supprimé,
        l'administrateur est apte à réaliser cela.
        \par
        Chaque action effectuée par un administrateur sera enregistrée automatiquement pour permettre la traċabilité
        et la non-répudiation\footnote{On abordera cette partie dans la section sécurité du chapitre 3.}.
        Ainsi, un module permettant de visualiser uniquement les logs\footnote{Historique des actions effectuées sur un système informatique.} du système. Par conséquent, on peut savoir
        la date et l'heure précise où un administrateur ouvre une session, affiche, ajoute, modifie ou supprime une donnée.
        Nul utilisateur ne pourra altérer ces donnéees.
        \par

        \paragraph{Le superadmin}
        Avant tout, il peut réaliser toutes les actions d'un visiteur. De plus, après s'être authentifié au moyen de 
        son adresse électronique et de son mot de passe, il peut interagir directement avec la base de données et 
        effectuer aussi les mêmes actions que l'administrateur.
        Dans ce module, le superadmin peut:
        \begin{itemize}
                \item \textbf{Ajouter un utilisateur: }
                Il s'agit de rentrer les informations relatives à un utilisateur pour l'ajouter dans la base de données.
                Ces informations sont de types différents (nom:texte, prénom identifiant:entier, type d'utilisateur, 
                etc\footnote{Les différents champs et leur type seront 
                détaillés dans l'étude des diagrammes à la fin du chapitre} )
                \item \textbf{Modifier un utilisateur: }
                Si pour une raison ou pour une autre les informations d'un utilisateur doivent être modifiées, le superadmin est en
                mesure de le faire après s'être authentifié. Un message lui sera affiché à l'écran dépendemment 
                de la réussite ou de l'échec de son action.
                \item \textbf{Activer ou désactiver un utilisateur: }
                Il s'agit d'autoriser ou non un administrateur à utiliser l'application.
        \end{itemize}
          

\par    
\begin{table}
        \centering
        \begin{tabular}{|p{0.21\linewidth}|p{0.54\linewidth}|p{0.21\linewidth}|}
        \hline
                \textbf{Utilisateurs} & \textbf{Besoins} & 
                \textbf{Contraintes}  \\
                \hline
                        Visiteur & 
                        \begin{itemize}
                                 \item[$\cdot$]  la cartographie d'Haïti
                                 \item[$\cdot$]  les fonds de carte
                                 \item[$\cdot$]  faire des recherches
                                 \item[$\cdot$]  de filtrer les donnéees
                                 \item[$\cdot$]  télécharger les résultats des tests
                                 \item[$\cdot$]  une navigation simple et attrayante
                        \end{itemize} & 
                         \\
                \hline
                        Administrateur & 
                        \begin{itemize}
                                \item[$\cdot$]  tous les besoins du visiteur
                                \item[$\cdot$]  ajouter un test
                                \item[$\cdot$]  modifier un test
                                \item[$\cdot$]  supprimer un test
                                \item[$\cdot$]  supprimer un commentaire
                        \end{itemize} & 
                        Authentification \\
                \hline
                        Superadmin & 
                        \begin{itemize}
                                \item[$\cdot$]  tous les besoins du visiteur
                                \item[$\cdot$]  tous les besoins d'un admin
                                \item[$\cdot$]  voir les logs
                                \item[$\cdot$]  ajouter un administrateur
                                \item[$\cdot$]  modifier un administrateur
                                \item[$\cdot$]  activer un administrateur
                                \item[$\cdot$]  désactiver un administrateur
                        \end{itemize} & 
                        Authentification \\
                \hline 
        \end{tabular}
        \caption{Tableau des utilisateurs et de leurs besoins} \label{tab:sometab}
\end{table}
\par