\subsection{Diagramme des cas d'utilisation}
    \subsubsection{Cas général}
    \paragraph{}
    \begin{figure}
        \centering
        \includegraphics[width=1\textwidth]{casGeneral.png}
        \caption{Diagramme des cas d'utilisation général}
    \end{figure}
    Trois niveaux d'acteurs sont à considérer au sein du système: un visiteur, 
    un administrateur et un super administrateur. La hiérarchisation permet que 
    chaque niveau ait accès aux droits du niveau immédiatement inférieur. 
    De ce fait, un super administrateur est également un administrateur et 
    un visiteur en plus de son niveau direct. \par 
Pour commencer, le visiteur a des droits d'accès très restreints: \par 
\begin{itemize}
    \item Parcourir le webmap: Le visiteur peut voir l'ensemble des informations 
    géotechniques disponibles sur la carte.
    \item Changer de fond de carte: Afin de mieux illustrer le contexte marquant 
    l'intérêt du visiteur, une variété de fonds de carte est accessible sur le site. 
    Ainsi, l'utilisateur peut puiser dans le champs de choix qui lui sont proposés.
    \item Visionner les données enregistrées: En cliquant sur une légende précise, 
    le visiteur peut voir les données qui ont été préalablement enregistrées dans la base de données.
    \item Effectuer des recherches: Deux options s'offrent aux utilisateurs. Ces derniers 
    peuvent afficher tous les résultats au cours de la rechercherche, ou encore ils peuvent 
    se fixer des filtres capables de mieux limiter les plages des résultats.
    \item Accéder aux résultats d'un test: Une fois les résultats obtenus, le visiteur 
    peut soit simplement les afficher, soit les imprimer.
\end{itemize}

\par 
De son côté, l'administrateur s'occupe de la gestion des informations au sein de la base 
de données. En plus des droits de visiteur, ce type d'utilisateur peut:
\begin{itemize}
    \item Ajouter des informations géotechniques: L'admin peut ajouter des informations 
    dans la bdd, qui sont reflétées sur la carte. 
    \item Modifier les informations qu'il avait préalablement enregistrées: Il ne peut 
    modifier que les informations qu'il avait lui-même ajoutées.
    \item Supprimer les informations qu'il avait préalablement enregistrées: Tout comme il 
    en est pour la modification, il ne peut supprimer que les informations qu'il avait lui-même ajoutées.
    \item Ajouter un fond de carte
    \item Modifier un fond de carte
    \item Supprimer un fond de carte
\end{itemize}

\par 
Évidemment, aucune de ces actions ne saura avoir lieu tant que l'administrateur ne s'est pas authentifié.

En dernier lieu, le super administrateur joue surtout un rôle de gestionnaire en ressources 
humaines. Une fois authentifié, en plus des droits d'accès d'un simple administrateur, 
cet utiliateur peut:
\begin{itemize}
    \item Ajouter des utilisateurs
    \item Modifier les utilisateurs
    \item Afficher les informations relatives aux différents utilisateurs, pouvant 
    ainsi retracer toutes les actions posées par un utilisateur du système.
    \item Supprimer ou désactiver un utilisateur: La différence se fait remarquer 
    par le fait que le super admin peut supprimer complètement un utilisateur ainsi 
    que toutes les informations y relatives ou simplement désactiver le compte d'un 
    utilisateur sans, pour autant, éliminer ses données.
\end{itemize}
