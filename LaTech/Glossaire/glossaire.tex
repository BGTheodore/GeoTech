\section{Glossaire}
\paragraph{URGéo :}
L'Unité de Recherche en Géosciences a pour mission de mener des
recherches dans les domaines des géosciences.
\par
\url{www.urgeo.net/}


\paragraph{BME}
Le Bureau des Mines et de l’Energie est un organisme autonome créé en 
1986 fonctionnant sous la tutelle du Ministère des Travaux Publics, Transports 
et Communications (MTPTC). 
\par
\url{www.bme.gouv.ht/}

\paragraph{SICOD}
La  Société d’Ingénierie Constructions et d’Orientations Diverses,
fondée en 2011, est une société haïtienne en noms collectifs qui évolue dans 
les domaines d’ingénierie géotechnique et de constructions.

\paragraph{LNBTP}
Le Laboratoire National Du Bâtiment et des Travaux Publics est une institution 
publique à gestion autonome chargée du contrôle de
la qualité des infrastructures en construction dans le pays. Il s'occupe 
aussi des études géotechniques, des recherches appliquées sur les matériaux de 
construction et de la promotion des normes en matière de génie civil.
\par
\url{www.lnbtp.gouv.ht/}


\paragraph{Géothechsol}
Géothechsol est un Bureau d’Etudes 
en Ingénierie Géotechni\-que et Environnemental
ainsi qu’en formulation de béton et ses essais mécani\-ques et physiques, qui s’est
fixé pour objectif de vous apporter une réponse sérieuse et de qualité, adaptée à 
vos besoins dans le respect de vos contraintes. 
\par
\url{www.geotechsol.com/}


\paragraph{SIG ou GIS :}  
Un système d'information géographique ou SIG (en anglais, Geographic 
Information System) est un système d'information conçu pour 
recueillir, stocker, traiter, analyser, gérer et présenter tous les 
types de données spatiales et géographiques. 



\paragraph{BRGM :}
Bureau de Recherches Géologiques et Minières


\paragraph{PDF :}
Portable Document Format. Le format PDF permet de conserver en 
toutes circonstances la mise en page 
originelle d'un document, quel que soit le logiciel ou le système 
d'exploitation utilisé pour l'ouvrir. Créé par la société Adobe, 
il est aujourd'hui très utilisé à travers le monde.

\paragraph{ISO :}
International Organization for Standardization.
L'Organisation internationale de normalisation généralement désigné sous son
 sigle : ISO, est un organisme de normalisation international composé de 
 représentants d'organisations nationales de normalisation de 164 pays.


\paragraph{CSV :}
Un fichier CSV (en anglais, Comma Separated Values) est le fichier de 
base des données recueillies - sans formatage particulier. Chaque 
champ est séparé par une virgule.

\paragraph{UI :}
L’UI design est directement lié à l’UX design. Cela signifie interface 
utilisateur. Il s’agit du lien direct entre l’utilisateur (le visiteur) 
et la machine (le programme ou la plateforme qui a permis de construire 
le site web). De nombreux éléments entrent dans l’UI design : la typographie, la police, la taille et la 
couleur d’écriture, les visuels, la charte graphique, identité visuelle, 
charte éditoriale, ou l’intuitivité.

\paragraph{UX :}
UX Design ou expérience utilisateur est un ensemble de techniques 
permettant de concevoir un site internet dans lequel le visiteur navigue 
de manière optimale. Le but est d’améliorer l’interaction entre l’homme 
et la machine. 

\paragraph{WGS84 :}
World Geodesic System (Sytème géodésique mondial) - révision de 1984
C'est un système de coordonnées terrestres, basé sur un géoïde de référence 
prenant la forme d'un ellipsoïde de révolution.
WGS84 est un système de coordonnées comprenant un modèle de la terre. Il est 
défini par un ensemble de paramètres primaires et secondaires :
les paramètres primaires définissent la forme de l'ellipsoïde de la terre, sa vitesse angulaire, et sa masse.
les paramètres secondaires définissent un modèle détaillé de la pesanteur terrestre.
